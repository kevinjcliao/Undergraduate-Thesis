\section{Introduction}
Programs crash. Static type systems serve to make the life of a programmer
easier by providing static compile-time guarantees of certain program behaviour,
guaranteeing that a function will map from a domain to a co-domain. A
dependently typed programming language gives a programmer even additional
expressive power, allowing types to be expressed relative to values given at
runtime. For example, consider a function \textit{replicate}, that takes in any
Peano natural number $n$ and returns a length-indexed vector of $n$ Peano
natural numbers of the value $n$. We can express this in Idris, a
dependently typed functional programming language, as follows in Figure
~\ref{replicate_in_idris}. 

\begin{figure}[ht!]
  \caption{Implementation of Replicate in Idris}
  \label{replicate_in_idris}
  \begin{lstlisting}
    -- replic 3 will give us (3 :: 3 :: 3 :: Nil)
    replic : (n : Nat) -> Vect n Nat
    replic n = replHelp n n where
      replHelp : (n : Nat) -> Nat -> Vect n Nat
      replHelp Z _ = Nil
      replHelp (S n) val = (val :: replHelp n val)
  \end{lstlisting}
\end{figure}

Thus, the co-domain of our function is dependent upon some value that is
provided at runtime. Calling \texttt{replic 5} will limit the co-domain of the
function to all length-indexed vectors of length 5, etc. Having written this
type, a programmer now has compile-time guarantees that the logic for a function
was expressed correctly and behaviour will correspond closer to what is
expected.

While research in dependent types has traditionally been in the domains of
theoretical mathematics, where researchers use dependently typed theorem provers
to write mathematical formalisms, dependent types have applications outside of
such a theoretical domain. In my work, I seek to find patterns in the existing
body of literature around practical applications of dependent types. My goal is
to enumerate signs that a programmer should look for that the solution to a
problem could benefit from the use of dependent types. 

\section{Work Covered}
This review is a tour of potential applications of dependent types. In this
abstract, I have chosen a brief example, units of measurement, in order to
demonstrate their potency. However, in addition to units of measurement,
dependent types have far ranging applications. A poster of my research would
contain a couple of the applications which I am exploring: 
\begin{enumerate}
  \item Complicated pattern matching in the Cryptol DSL for cryptographic
  applications. \cite{power_of_pi}
  \item Generating file format parsers from a data description language.
  \cite{power_of_pi}
  \item A relational database and algebra. \cite{power_of_pi, eisenberg2016}
  \item An alternative to monadic transformations with an algebraic effects DSL
  \cite{algebraic}
  \item Programming distributed systems with F* and dependent types.
  \cite{fstar_distributed_programming}
  \item A low-level domain-specific language demonstrating the applicability of
  dependent types to systems programming. \cite{idris_systems_programming}
\end{enumerate}

\section{Units of Measurement}
\textit{This section is adapted from Adam Gundry's Ph.D Dissertation
\cite{gundry2013}, among other resources for well-typed unit measurements with
dependent types.}

My first example is to demonstrate how dependent types can eliminate bugs that
can arise from improper unit conversions \cite{gundry2013}. Units of measurement
are already implemented in Microsoft's F\# Programming Language
\cite{kennedy2009}. If numbers carry a type denoting their unit of measurement
with them, we can ensure at compile time that improper unit conversions are not
going to occur at runtime. These bugs can be catastrophic, as made evident by
NASA's loss of the \$125-million Mars Climate Orbiter when ``spacecraft
engineers failed to convert from English to Metric units of measurement''
\cite{hotz1999}. 

\begin{figure}[ht!]
  \caption{Displays a unit error that would be caught at compile-time with units of measurement.}
  \label{unit_error}
  \begin{lstlisting}
    distanceTraveled : Quantity Kilograms
    distanceTraveled = inches 20

    distanceLeft : Quantity Metres
    distanceLeft = (metres 1000) - distanceTraveled
  \end{lstlisting}
\end{figure}

An implementation of units of measurement should have types that support
decidable equality by definition. Two typed variables can only be equal because
they have the same unit of measurement or derived unit of measurement and the
same value. This means that if coding style guidelines enforce that all numeric
values must be well-typed with units of measurement, there will be compile-time
guarantees that errors of conversion between units of measurement will not
occur. See Figure ~\ref{unit_error} for an example of a program that should
error \footnote{This example should error in the case we've described here,
where quantities are indexed by their \textit{unit} and not their
\textit{dimension}, however, if the latter is done, a compiler could do a
conversion implicitly. This is explored in \cite{eisenberg2014}.}

While units of measurement are implemented as a feature in the F\# language,
which is not dependently typed, a dependently typed programming language would
allow for a units of measurement system to be implemented \cite{gundry2013}.
Gundry invites us to consider a system for describing units in terms of a
constructor that allows us to both enumerate elementary units and also express
derived units in terms of one another \cite{gundry2013}. 

\begin{figure}[ht!]
  \label{idris_code}
  \caption{Basic SI unit declarations in adapted from Dependent Haskell to Idris \protect\cite{gundry2013}}
  \begin{lstlisting}
    data Unit : Int -> Int -> Int -> Type
    
    Dimensionless : Type
    Dimensionless = Unit 0 0 0
    
    Metres : Type
    Metres = Unit 1 0 0
    
    Seconds : Type
    Seconds = Unit 0 1 0
    
    Kilograms : Type
    Kilograms = Unit 0 0 1
    
    data Quantity u = Q Double
    
    metres : Double -> Quantity Metres
    metres v = (Q v)
    
    seconds : Double -> Quantity Seconds
    seconds v = (Q v)
    
    kilograms : Double -> Quantity Kilograms
    kilograms v = (Q v)
    
    plus : Quantity u -> Quantity u -> Quantity u
    plus (Q x) (Q y) = Q (x + y)
  \end{lstlisting}
\end{figure}

For now, our data declaration for \texttt{Unit} only supports three elementary
units (metres, seconds, kilograms), but one can imagine a full library
implementing the entire SI Units system. Each elementary unit is implemented as
a single 1 in the call to the Unit constructor with all entries as zero. Thus,
we can express derived units in a call to the Unit constructor where negative
integers would represent elementary units present in the denominator. 

We can define quantities as a type containing a \texttt{Unit} and an integer.
This then allows us to write simple constructors for the quantity type. We can
then define well-typed multiplication and addition operations giving us similar
guarantees to that which is given by units of measurement in F\#. 

As defined above, this enforces well-typed addition, requiring that two
values be of the same type. we can also define operations that allow us to
express fractional units. The types for \texttt{times} and \texttt{over}
representing division and multiplication are shown in Figure~\ref{division}. In
this code, we see that multiplying or dividing two units together at the
value-level does an addition operation on all the elementary units at the type
level. For example, a Newton of force is defined as a $kg\times ms^{-2}$.
Therefore, if we are able to compose types through multiplication and division,
we can express a Newton with our units system. See Figure ~\ref{division}. 

\begin{figure}[h]
  \caption{Definition of division and multiplication of dependently typed units
  of measurement. Ported to Idris from \protect\cite{gundry2013}}
  \label{division}
  \begin{lstlisting}
    times : Quantity (Unit m s g) -> Quantity (Unit m' s' g') 
            -> Quantity (Unit (m + m') (s + s') (g + g'))
    times (Q x) (Q y) = Q (x * y)

    inverse : Quantity (Unit m s g) -> Quantity (Unit (-m) (-s) (-g))
    inverse (Q x) = Q (1 / x)

    over : Quantity (Unit m s g) -> Quantity (Unit m' s' g')
            -> Quantity (Unit (m -m') (s - s') (g - g'))
    over x y = times x (inverse y)

    Newtons : Type
    Newtons = Unit 1 -2 1

    newtons : Double -> Quantity Newtons
    newtons val = over 
        (times (kilograms val) (metres 1)) 
        (times (seconds 1) (seconds 1))
  \end{lstlisting}
\end{figure}

What we've shown here is that while units of measurement can be first-class
features in a programming language like F\#, a dependently typed language allows
us to build certain functionality easily into the language without changing the
language specification whatsoever. This system is merely a simplistic
proof-of-concept designed to demonstrate to the user that dependently-typed
languages can implement units-of-measurement. By allowing integer values to
appear as part of a type signature, we've written a rudimentary system that
allows a user to write well-typed units. In addition, the user is able to define
derived units (see our definition of Newton). 

Since derived units have become type-level arithmetic, we know that we will be
able to consistently derive units where required in contrast to other methods in
non-dependently-typed programming languages. Our implementation is able to
derive units for examples that F\# is unable to \cite{gundry2013,f_sharp_units}.
At the same time, error messages remain very difficult to debug. Rather than
showing the unit names for conversions that fail, Idris will show the type
declaration (e.g. Instead of showing \texttt{Metres} in an error message, Idris
will show \texttt{Unit 1 0 0}). 

\section{Conclusions}
Through a review of the practical applications of dependent types in existing
and implemented functional programming languages, I demonstrate that while
dependent types are often thought of as far-off and theoretical, they are
currently available in advanced functional programming languages and serve
practical purposes. 

I set out to look for patterns that signal to a programmer that dependent types
would be particularly useful for their work. So far, I've found that the
problems that I'm exploring seem to be focused around two main themes: the
serializing and deserializing of data, and building domain-specific languages.
As I continue my research, I will expand on these ideas and make the scenarios
more concrete and identifiable to a programmer, which I hope to present at POPL.