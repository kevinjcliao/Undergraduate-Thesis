
\chapter{Literature Review}

\section{Background}
A dependently typed programming language can have functions with types that
depend on a value. A function, at its core, is a map from a domain to a
co-domain. In other words, we expect there to be a certain set of elements in
the universe for which our function can give us a corresponding output. A way to
remove certain bugs in programs is to ensure that a function in a program is
indeed mapping from the correct set of potential inputs to the set of potential
outputs. 

One can consider static type systems as a way to narrow down the set of
potential inputs to the set of possible outputs. For example, a function that
takes in a string and outputs an integer gives certain compile-time guarantees
to its programmer. If compilation succeeds, the domain of this function will be
strictly limited to an element in the set of all possible strings in the
universe and the output will be limited to an element of the set of all possible
integers. 

However, consider, for example, a function that appends an item to a list. Under
a regular type system, we would say that this function takes in a list of
elements of type a, an element of type a, and returns elements of type a. A
Haskell type signature for this function would look like this: 

\texttt{append :: [a] -> a -> [a]}. 


Let's imagine that we have a list data type signature that contains information
not only about the type of the elements that the list contains, but also about
the length of the list. That is to say, the type signature of a vect (list with
length-in-type) can be expressed as: 

\begin{figure}
    \caption{Using a Vect data type.}
    \begin{lstlisting}
        Vect :: Int -> Type -> Type -- A list has an integer denoting length, 
        -- and the type of its elements. [1,2,3] :: Vect 3 Int
    \end{lstlisting}
\end{figure} 


Now that we've introduced the length of the vect type as part of its type
signature, we can write a much more strict and bug-free type signature for our
append function. Essentially, any append function would take any vect with
length $n$ and type $a$. It also takes in an element of type $a$ to append. It
outputs a list of length $n+1$ and type $a$. This type signature looks like: 

$$
\texttt{append :: Vect n a -> a -> Vect (n+1) a} 
$$

What's peculiar about this is that the co-domain of this function is not
particularly fixed. In fact, it depends on the value of its input. For example,
if a list of length 3 and type Int is inputted, the co-domain of our function is
the set of all lists with length 4 and type integer. This is an example
application of dependent types. What we've done is created a function where the
co-domain varies as the input value varies. The guarantee of type safety
provided by this type signature is substantial. 

The goal of dependent types is to write programs with extreme guarantees of
compile-time safety. We can use the types of the parameters of a function to
place tighter limits on the set that consists of its co-domain, with the
co-domain varying depending on the values of the input parameters. 

In this literature review, I will explore existing literature around practical
real-world applications of dependent types. I'll take a look at three examples
where a domain specific language can be built if a language can support full
dependent types. I'll then show how dependent types can be applied to make
systems programming and building distributed systems safer. I'll also take a
look at how a dependently typed language can implement units of measurement,
preventing a set of potentially costly and fatal human errors. The hope is to
demonstrate that dependent types, long confined to theoretical mathematics, have
tremendous promise in helping programmers build reliable and safe programs. 

\section{Dependent Types: A History}
Dependent types in programming languages have their roots in intuitionistic type
theory or Martin-L\"{o}f Type Theory \cite{intro_martin_lof, ml_type_theory}. This
type theory serves as a foundation for \textit{constructive mathematics}
\cite{constructive_math}. Per Martin-L\"{o}f was interested in a type theory that
could be used as a programming language, where all well-typed programs must
terminate \cite{ml_type_theory}. His type theory is based on the principles of
constructive mathematics, mainly Curry-Howard isomorphism. Curry-Howard
isomorphism is the notion that there is a direct correspondence between
mathematical proofs and type theory. In other words, a type signature is
synonymous with a mathematical proposition, and if a valid program satisfying
the constraints of such a type signature exists, then a proof proving the
corresponding mathematical proposition \cite{martin_lof, ml_type_theory}. To
illustrate this, consider the examples in Figure ~\ref{ml_type_theory_examples}
\footnote{These examples were provided by Prof. Richard Eisenberg in a
discussion.}. 

\begin{figure}[h]
    \label{ml_type_theory_examples}
    \caption{Comparison between Haskell type signatures and mathematical proofs to illustrate Curry-Howard Isomorphism}
    \begin{tabular}{c|c|c|c}
        Haskell Type Signature & Math Proposition & Haskell type signature inhabited? & Proof exists? \\
        $\forall a. a \rightarrow a$ & $p \rightarrow p$ & True & True \\
        $\forall ab. (a,b) \rightarrow a$ & $ (p \wedge q) \rightarrow p$ & True & True \\
        $\forall ab. a \rightarrow b$ & $p\rightarrow q$ & False & False \\
        $\forall ab. a \rightarrow (a,b)$ & $p \rightarrow (p\wedge q)$ & False & False 
    \end{tabular}
\end{figure}

From the figure, we see the Curry-Howard isomorphism that forms the fundamentals
of Martin-L\"{o}f's type theory. We see that in all four cases, if there is a
function that inhabits the type signature, the mathematical proposition
presented is valid and if a function does not exist, the proposition is invalid. 

Mathematicians took an interest in creating a programming language based on
Curry-Howard isomorphism and Martin-L\"{o}f type theory, since Curry-Howard
isomorphism meant that a valid function that inhabits a type signature could be
equivalent to a proof. A dependently typed programming language based on
foundations in Martin-L\"{o}f Type Theory called NuPrl was first released in
1984 \cite{nuprl}. NuPrl is used as a \textit{proof assistant} that helps
mathematicians and programmers formalize proofs \cite{prl_home}. Dependently
typed proof assistants like NuPrl found a home at the intersection between
programming language enthusiasts interested in total program correctness and
constructive mathematicians interested in systems where mathematical formulisms
could be systematically encoded. Other proof assistants with support for
dependent types followed: Coq (1989) \cite{girard1989proofs}, ALF (1990)
\cite{Magnusson95theimplementation}, Cayenne (1998)
\cite{Augustsson:1998:CLD:291251.289451}, Agda (1999)
\cite{structured_type_theory}.

While there are now robust theorem provers that incorporate dependent types,
work now is primarily concerned with bringing them into mainstream programming
and software development. Idris (2011) was designed with general purpose
programming in mind \cite{tdd_book}. F* (2011) was introduced by Microsoft as a
dependently typed language specifically designed around solving problems in
secure distributed programming \cite{fstar_distributed_programming}. In addition
to the development of new programming languages with dependent type systems
built into the language by design, work exists to mainstream dependent types
into more prominent programming languages. The most active and promising
mainstreaming work is on Haskell \cite{eisenberg2016, gundry2013}. 

\section{Implementing Domain-Specific Languages with Dependent Types}

\subsection{Cryptol: A DSL for Cryptography}

Dependent type systems have potential applications in easily implementing domain
specific languages (DSLs). Cryptol, for example, is a domain-specific language
designed around cryptography (\cite{cryptol_manual}). Problems inherent in
implementing a Cryptol compiler or interpreter can be solved through dependent
types (\cite{power_of_pi}). Cryptol is a functional programming language with
advanced support for pattern matching. Since cryptography commonly requires
dealing with low-level bit manipulation, it follows that Cryptol is designed
around facilitating these operations and making them safe. A function that does
this sort of low-level manipulation is the \texttt{swab} function, which takes
in a 32-bit word and swaps the first two bytes (\cite{cryptol_manual}).: 

$$
\texttt{swab :: Word 32 -> Word 32} \\
\texttt{swab [a b c d] = [b a c d]} $$

Ideally, a word would be represented by a vector of 32-bits. We would be able to
declare a pattern match with swab that looks similar to the declaration
presented by Oury and Swiestra above. How then does the compiler understand that
this pattern match declaration means we expect the input vector to be divided
into 4 separate vectors of 8 bits? This is where dependent types serve a
practical purpose. By specifying types that split the length of the vector up
into a multiple of two scalars, we can effectively implement this clever pattern
match, allowing for powerful pattern matching required by the Cryptol language
(\cite{power_of_pi}). 

\textit{More coming in final draft of literature review}

\subsection{PBM: Generating Parsers with Data Description Languages}

Work also exists to use dependent types in creating embedded data description
languages, which are languages where a programmer can describe the structure of
data and quickly generate a working parser \cite{power_of_pi}.  For example,
consider the portable bitmap (pbm) file format, which consists simply of ``P4'',
followed by the dimensions of the image in pixels as $n, m$ integers separated
by a space. After a newline, hde image is described as a string of $n\times m $
bits where 1 is black and 0 is white \cite{pmb_spec}. If a parser were generated
from a data description language, we expect the parser to either return
well-typed data (a vector of bits and the dimensions of the image) or to signal
that the data is not well-formed. In other words, if we want to generate parsers
through embedded data description languages, we could specify the file format as
a value. The type of the generated parser then, would depend on the file format
as a value, making this an appropriate area to apply dependent types
\cite{power_of_pi}. 

We can start by defining our \textit{universe} (see Figure ~\ref{universe}),
which contains all the types that our parser will be manipulating in some way.
We also define a function \textit{el}, which will take any value of type U and
convert it to an appropriate type. The combination of this data type declaration
and this \textit{el} function is a definition of the relevant universe for this
problem domain \cite{power_of_pi}. 

\begin{figure}[h]
    \caption{Universe declaration in Idris \cite{power_of_pi}, Idris
    implementation by \cite{idris_pop}.}
    \label{universe}
    \begin{lstlisting}
        data Bit : Type where 
            O : Bit 
            I : Bit

        data U : Type where
            STRING : U
            BOOL : U
            CHAR : U
            NAT : U
            VECT : Nat -> U -> U
        
        el : U -> Type
            el STRING     = String
            el BOOL       = Bool
            el CHAR       = Char
            el NAT        = Nat
            el (VECT n u) = Vect n (el u)
    \end{lstlisting}
\end{figure}

From here, we can define a \textit{Format} data type that enables us to describe
the format of our data. When sequencing formats, we want two binary operators
that either read or skip. To skip means to skip over the first parameter and
generate a type for the file format from the second parameter. To read means to
build a type from both parameters before moving on. We will need to define both
these operations, a base operation that gives us a type, a terminal, and
rejection if the input data is badly formed. We can declare such a type as
follows: 

\begin{figure}[h]
    \caption{Format data type in Idris \cite{power_of_pi}, Idris implementation
    by \cite{idris_pop}.}
    \label{formatDeclaration}
    \begin{lstlisting}
        data Format : Type where Bad : Format End : Format Base : U -> Format
        Plus : Format -> Format -> Format Skip : Format -> Format -> Format Read
        : (f : Format) -> (Fmt f -> Format) -> Format

        Fmt : Format -> Type Fmt Bad = Void Fmt End = Unit Fmt (Base u) = el u
        Fmt (Plus f1 f2) = Either (Fmt f1) (Fmt f2) Fmt (Read f1 f2) = (x : Fmt
        f1 ** Fmt (f2 x)) Fmt (Skip _ f) = Fmt f

        (>>) : Format -> Format -> Format f1 >> f2 = Skip f1 f2

        (>>=) : (f : Format) -> (Fmt f -> Format) -> Format x >>= f = Read x f
    \end{lstlisting}
\end{figure}

In the code of Figure ~\ref{formatDeclaration} one thing noteworthy is the
$(**)$ operator, which is Idris syntactic sugar for a dependent pair. A
dependent pair (a : A ** P) means that the type variable $a$ is of type A and
can also occur in the type P \cite{idris_dependent_pairs}. For example, consider
Figure ~\ref{dependentPairExample}. The example will only type check correctly
iff the natural number present in the first element of the pair is the same as
the length of the list. 

Having specified a data type that lets us declare formats, we can then move on
to creating a specification. In Figure ~\ref{spec_declaration}, a format for the
PBM spec is provided. We are now able to write a parser that takes in a Format
as a data type, and then is able to parse files to generate well-typed data. See
Figure ~\ref{parser}. This straightforward parsing code is aided by the types
that we declared before, skipping, reading, and terminating where required by
our file format specification. We can use Idris' REPL to see how the parser
deals with our PBM specification in Figure ~\ref{repl}. Here, we see that the
type signature of the function created by giving the parse function our pbm
specification is a function that takes in a list of bits and returns a matrix of
bits with sizes bound by the natural numbers that we first specified in the file
format. 

\begin{figure}[h]
    \caption{Example for Dependent Pairs taken from the Idris documentation.}
    \label{dependentPairExample}
    \begin{lstlisting}
        vec : (n : Nat ** Vect n Int) vec = (2 ** [3, 4])
    \end{lstlisting}
\end{figure}

\begin{figure}[h]
    \caption{Format declaration of PBM \cite{power_of_pi}, Idris implementation
    by \cite{idris_pop}.}
    \label{spec_declaration}
    \begin{lstlisting}
        export pbm : Format pbm = char 'P' >> char '4' >> char ' ' >> Base NAT
        >>= \n => char ' ' >> Base NAT >>= \m => char '\n' >> Base (VECT n (VECT
        m BIT)) >>= \bs => End
    \end{lstlisting}
\end{figure}


\begin{figure}[h]
    \caption{Parser declaration \cite{power_of_pi}, Idris implementation by
    \cite{idris_pop}.}
    \label{parser}
    \begin{lstlisting}
        parse : (f : Format) -> List Bit -> Maybe (Fmt f, List Bit)
        parse Bad bs       = Nothing
        parse End bs       = Just ((), bs)
        parse (Base u) bs  = read u bs
        parse (Plus f1 f2) bs with (parse f1 bs)
          | Just (x, cs)   = Just (Left x, cs)
          | Nothing with (parse f2 bs)
            | Just (y, ds) = Just (Right y, ds)
            | Nothing      = Nothing
        parse (Skip f1 f2) bs with (parse f1 bs)
          | Nothing        = Nothing
          | Just (_, cs)   = parse f2 cs
        parse (Read f1 f2) bs with (parse f1 bs)
          | Nothing        = Nothing
          | Just (x, cs) with (parse (f2 x) cs)
            | Nothing      = Nothing
            | Just (y, ds) = Just ((x ** y), ds)
    \end{lstlisting}
\end{figure}

\begin{figure}
    \caption{Putting the PBM spec into Idris' REPL}
    \label{repl}
    \begin{lstlisting}
        *Parser> :t parse pbm parse pbm : List Bit -> Maybe ( (x : Nat ** x1 :
        Nat ** x2 : Vect x (Vect x1 Bit) ** ()) , List Bit)
    \end{lstlisting}
\end{figure}

In this section, we show that dependent types allow us to create embedded data
description languages inside of a dependently typed language. We can then
generate well-typed, reliable parsers without having to rewrite a lot of code.
Thus, using dependently typed languages to write parsers with embedded data
description languages both demonstrates promise in brevity and also safety. 

\subsection{Safer Databases: Relational Algebras}

Databases are one of the pillars of modern software. Databases are employed in
everything from social media software \cite{tao} to flight scheduling and
booking \cite{flights}. Different programming languages have varying interfaces
for querying an SQL database. Oftentimes, a query and response interface simply
asks a user to send and receive a SQL query and as a string and to process a
response given as a string. While this offers flexibility, it means that there
is no compile-time guarantee of correctness and that poorly written SQL queries
result in run-time crashes, rather than compile-time debugging. We would like to
be able to easily compose types from a set of values (database schema), which
would allow a programmer to easily compose type-safe queries with compile-time
verification of correctness \cite{power_of_pi}. Many SQL interfaces do runtime
type checking, meaning if we can avoid these type checks, there may be
performance gains. 

\begin{figure}[h]
    \caption{Declaration of schema. Idris adaption of code from Power of Pi
    \cite{power_of_pi}}
    \label{schema}
    \begin{lstlisting}
        Attribute : Type}
        (String, U)}

        Schema : Type}
        List Attribute}

        Students : Schema}
        Students = (   ("name", STRING)}
                    :: ("id", VECT 6 CHAR)} 
                    :: ("classyear", VECT 4 CHAR)}
                    :: Nil)}

        data Row : Schema -> Type where
            EmptyRow : Row Nil
            ConsRow : el u -> Row s -> Row ((name, u) :: s)

        kevin : Row Students
        kevin = ConsRow "Kevin Jiah-Chih Liao"
            (ConsRow ('1'::'2'::'3'::'4'::'5'::'6'::Nil)
            (ConsRow ('2'::'0'::'1'::'8'::Nil) EmptyRow))
    \end{lstlisting}
\end{figure}

We can start by defining types to represent a database schema. A database table
is a row of elements that correspond to a declared schema. A schema is a list of
attributes that each element should have. An attribute is simply a column name
and the type of what the column contains. Thus, we can declare an attributes and
schemas as follows in Figure ~\ref{schema}. 


Where U refers to the universe that we built in an earlier example. Given that
we built a schema, we are now able to define a schema to hold students. Take,
for example, a schema that stores a student's name, student id, and class year.
Our next job is to be able to express a database table with a row of instances
of a schema. This is reflected in our declaration of the \texttt{Row} type,
which lets us join rows together, ending in an EmptyRow. An example
\texttt{kevin} is provided. 

\begin{figure}[h]
    \caption{Declaration of connect. Idris adaption of code from Power of Pi
    \cite{power_of_pi}}
    \label{connect}
    \begin{lstlisting}
        -- Connect takes in a server name, a table name, 
        -- and a schema, and executes IO operations on that table. 
        -- This type ensures that upon connection, if successful, 
        -- the table on the server must have the same schema. 
        connect : String -> String -> (s : Schema) -> IO (Handle s)

        
        -- Only modeling a single 'read' operation. 
        data RA : Schema -> Type where
            Read : Handle s -> RA s
        
        -- Takes in a query written with our relational algebra and then sends back a table that corresponds to our schema. 
        toSQL : RA s -> IO (List (Row s))
    \end{lstlisting}
\end{figure}

Now that we've defined the schema and tables, it's time to show how we connect
and query the table. Usually this means that we have a function \texttt{connect}
that takes in a servername and tablename as strings, and a SQL query as a
string, before returning a string in an IO monad. Now that we've defined a
schema type, we can build a well-typed database connection that validates that
the database table we are connecting to is of the same schema as the schema we
are requesting. If connection succeeds, that means that all subsequent requests
with the \texttt{Handle schema} type returned by the connect function must be
safe, because we know that the schemas we are reading or writing from the table
must correspond to the schema in our code. This connect function is defined in
Figure ~\ref{connect}. 

Finally, we define a type "Relational Algebra", which contains the opperations
that we would perform in a database query. For the sake of brevity, we've 
limited it to \texttt{Read}. We then have a function that does an IO operation
on a database, taking in a query of type Relational Algebra, to return us
a database table of consisting of the results of our query. 

Thus, in this example, we've shown the potential for dependent types to build a
type safe database, where if the schemas match on connection, all subsequent
queries should be type safe. While there was some overhead in setting up the
types, once the database software is written, declaring a schema with elements
in our universe was pretty easy and I anticipate database queries not being
significantly more difficult to compose and write. Thus, this is an exciting
application of dependent types where programmers can potentially benefit from
high-level use of dependently typed software once the more complex moving parts
are written. 




\section{Systems Programming with Dependent Types}

General purpose dependently typed programming languages such as Idris (and in
the future, Haskell), allow programmers to integrate dependent types into lower
level work than a theorem proving language would allow. This section of the
literature review will take a look at Brady's \textit{Idris: Systems Programming
Meets Full Dependent Types} \cite{idris_systems_programming}.

\section{Well-Typed Unit Measurements with Dependent Types}

A practical application for dependent types from Gundry's thesis is to eliminate
bugs that can arise from improper unit conversions. Units of measurement are
already implemented in Microsoft's F\# Programming Language
(\cite{kennedy2009}). If numbers carry a type denoting their unit of measurement
with them, we can ensure at compile time that improper unit conversions are not
going to occur at runtime. These bugs can be catastrophic, as made evident by
NASA's loss of a\$125-million ``Mars Climate Orbiter'' when ``spacecraft
engineers failed to convert from English to Metric units of measurement''
\cite{hotz1999}. 

\begin{figure}
  \caption{Displays a unit error that would be caught at compile-time with units of measurement.}
  \label{unit_error}
  \begin{lstlisting}
    distanceTraveled : Quantity Inches
    distanceTraveled = inches 20

    distanceLeft : Quantity Metres
    distanceLeft = (metres 1000) - distanceTraveled
  \end{lstlisting}
\end{figure}

An implementation of units of measurement should have types that support
decidable equality by definition. Two typed variables can only be equal because
they have the same unit of measurement or derived unit of measurement and the
same value. This means that if coding style guidelines enforce that all numeric
values must be well-typed with units of measurement, there will be compile-time
guarantees that errors of conversion between units of measurement will not
occur. See Figure ~\ref{unit_error} for an example of a program that should
error. 

While units of measurement are implemented as a feature in the F\# language,
which is not dependently typed, a dependently typed programming language would
allow for a units of measurement system to be implemented \cite{gundry2013}.
Gundry invites us to consider a system for describing units in terms of a
constructor that allows us to both enumerate elementary units and also express
derived units in terms of one another. 

\begin{figure}[h]
  \label{idris_code}
  \caption{Basic SI unit declarations in adapted from Dependent Haskell to Idris \cite{gundry2013}}
  \begin{lstlisting}
    data Unit : Int -> Int -> Int -> Type
    
    Dimensionless : Type
    Dimensionless = Unit 0 0 0
    
    Metres : Type
    Metres = Unit 1 0 0
    
    Seconds : Type
    Seconds = Unit 0 1 0
    
    Kilograms : Type
    Kilograms = Unit 0 0 1
    
    data Quantity u = Q Double
    
    metres : Double -> Quantity Metres
    metres v = (Q v)
    
    seconds : Double -> Quantity Seconds
    seconds v = (Q v)
    
    kilograms : Double -> Quantity Kilograms
    kilograms v = (Q v)
    
    plus : Quantity u -> Quantity u -> Quantity u
    plus (Q x) (Q y) = Q (x + y)
  \end{lstlisting}
\end{figure}

For now, unit only supports three elementary units (metres, seconds, kilograms),
but one can imagine a full library implementing the entire SI Units system. Each
elementary unit is implemented as a single 1 in the call to the Unit constructor
with all entries as zero. Thus, we can express derived units in a call to the
Unit constructor where negative integers would represent elementary units
present in the denominator. 

We can define quantities as a type containing a \texttt{Unit} and an integer.
This then allows us to write simple constructors for the quantity type. We can
then define well-typed multiplication and addition operations giving us similar
guarantees to that which is given by units of measurement in F\#. 

As defined above, this enforces well-typed addition, requiring that two
additions be of the same type. we can also define operations that allow us to
express fractional units. For example, a Newton of force is defined as a
$kg\times ms^{-2}$. Therefore, if we are able to compose types through
multiplication and division, we can express a Newton with our units system. See
Figure ~\ref{division}. 

\begin{figure}[h]
  \caption{Definition of division and multiplication of dependently typed units
  of measurement. Ported to Idris from \cite{gundry2013}}
  \label{division}
  \begin{lstlisting}
    Newtons : Type
    Newtons = Unit 1 -2 1

    newtons : Double -> Quantity Newtons
    newtons val = over 
        (times (kilograms val) (metres 1)) 
        (times (seconds 1) (seconds 1))
  \end{lstlisting}
\end{figure}

What we've shown here is that while units of measurement can be first-class
features in a programming language like F\#, a dependently typed language allows
us to build certain functionality easily into the language without changing the
language specification whatsoever. 

\section{Programming Distributed Systems}

To be included in the final literature review. A review of ``Secure Distributed
Programming with Value-Dependent Types''. \cite{fstar_distributed_programming}

\section{Programming with Algebraic Effects}

In functional programming, we want to isolate side effects as much as possible
to keep our code clear. In Haskell and many other programming languages, side
effects like IO, State, Random Number Generation, etc. are handled by Monads
<CITATION NEEDED>. If we want to use several Monads at once (our code requires
simultaneous handling of different side effects), we are often required to use
monad transformers. While this approach works for programs that require one or
two transformations between monads, as we bring in more and more side effects,
the number of transformation monads we need to write increases quite quickly. 

Work exists to sidestep the problem of handling increasingly complex monadic
transformations by encoding algebraic effects as a domain-specific language in a
dependently typed programming language \cite{algebraic}. We can start by
defining an \texttt{EFFECT} type, as seen in Figure ~\ref{effects_def}. In order
for a function to use our effects DSL, it will have to be of type \texttt{Eff},
where Eff is a data declaration where the 'execution context' m (optionally a
Monad) is specified, a list of side effects, and the program's return type. For
example, a function that of the execution context \texttt{IO} that throws side
effects, does work on STDIO, and maintains an integer state will look something
like the function \texttt{example} in Figure ~\ref{effects_def}. 

\begin{figure}[h]
    \caption{Definition of effect type}
    \label{effects_def}
    \begin{lstlisting}
        data EFFECT : Type where
            STATE     : Type -> EFFECT
            EXCEPTION : Type -> EFFECT
            FILEIO    : Type -> EFFECT
            STDIO     : EFFECT
            RND       : EFFECT
        
        data Eff : (m : Type -> Type) -> (es : List EFFECT) -> 
            (a : Type) -> Type
    
        example : Eff IO [EXCEPTION String, STDIO, STATE Int] ()
    \end{lstlisting}
\end{figure}

We can now apply this small Effects DSL we have defined to work on some simple
programs where we need to maintain a side effect of some sort. I will provide an
example of a program where we tag each node of a binary tree with a unique ID. 

\begin{figure}[h]
    \caption{Tagging a binary tree with integers. Taken from Brady's work.
    \cite{algebraic}}
    \label{tag_def}
    \begin{lstlisting}
        -- Simple type def of binary tree in Idris
        data Tree a = Leaf
                    | Node (Tree a) a (Tree a)
        
        -- Takes in a tree and produces a tagged tree with
        -- State containing an integer passed inside of
        -- the function. 
        tag : Tree a -> Eff m [STATE Int] (Tree (Int, a))
        tag Leaf = return Leaf
        tag (Node l x r) = do
            l' <- tag l
            lbl <- get; put (lbl + 1)
            r' <- tag r
            return (Node l' (lbl, x) r')
        
        get : Eff m [STATE x] x
        
        put : x -> Eff m [STATE x] ()
        
        EffM :  (m   : Type -> Type) ->
                (es  : List EFFECT) ->
                (es' : List EFFECT) ->
                (a   : Type) -> Type
        
        runPure : Env id es -> EffM id es es' a -> a
        
        tagFrom : Int -> Tree a -> Tree (Int, a)
        tagFrom x t = runPure [x] (tag t)
    \end{lstlisting}
\end{figure}

\section{Proposal for Future Work}

While I'm still uncertain about the direction to proceed, I'm interested in
looking at elections and e-voting and whether or not we can provide guarantees
of correctness to vote counting software written in dependently typed languages.
I take a particular interest in the Australian Senate voting verification
process because verification of vote count is an NP-complete problem
(\cite{aus_senate}). 

If we were able to verify that vote counting software is correct at
compile-time, we would sidestep the need to run verification code that is trying
to solve an np-complete problem. Currently, the Australian government uses
proprietary code to count Australian senate ballots and has refused to release
the source code after a Freedom of Information Act request
(\cite{aus_senate_news}). If an open-sourced, verifiably correct counting
program were devised, we could greatly protect the integrity of Australian
elections. 

\section{Conclusion}
While many literature reviews begin by looking examining a problem and looking
for existing solutions, this literature review takes an opposite approach. The
broader problem we are trying to answer is one that crosses various domains and
engineering fields. To put it quite simply, \textit{programs crash}. Dependent
typed languages, long a toy for theoretical computer scientists and
constructivist mathematicians, are increasingly becoming realistic tools to
write code with necessary guarantees of correctness. In other words, dependent
types are a solution in search of a problem. 

In this literature review, I offered a brief summary as to what dependent types
are and what languages exist where dependent type functionality is available. I
then moved on to describe different applications of dependently typed
programming that exist in literature. I started by looking at Cryptol, a DSL for
cryptography, and showed how dependent types allow for implementing complex
pattern-matching that the language requires \cite{power_of_pi}. I then moved on
to discuss embedded data description languages, showing how one can describe how
data is structured and generate a parser out of such a description
\cite{power_of_pi}. I also examined the potential of dependent types to build a
typesafe database, eliminating runtime typechecking and thus reducing error and
increasing performance \cite{power_of_pi, eisenberg2016}. 

Outside of domain specific languages, I also showed the application of dependent
types to systems programming \cite{idris_systems_programming}, building
distributed systems \cite{fstar_distributed_programming} and units of
measurement \cite{gundry2013}. In this wide-ranging review, I've demonstrated
that as dependent types become brought into the mainstream, they have the
potential to empower programmers to build safe, robust programs in ways that
have not been possible before. 