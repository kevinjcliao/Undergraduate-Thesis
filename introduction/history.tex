\section{History}

Dependent types in programming languages have their roots in intuitionistic type
theory or Martin-L\"{o}f Type Theory \cite{intro_martin_lof,ml_type_theory}.
This type theory serves as a foundation for \textit{constructive mathematics}
\cite{martin_lof}. Per Martin-L\"{o}f was interested in a type theory that could
be used as a programming language, where all well-typed programs must terminate
\cite{ml_type_theory}. His type theory is based on the principles of
constructive mathematics and is interesting because of the Curry-Howard
isomorphism. The Curry-Howard isomorphism is the notion that there is a direct
correspondence between mathematical proofs in constructive mathematics and
programs. In other words, a type signature is synonymous with a mathematical
proposition, and if a valid program satisfying the constraints of such a type
signature exists, then it is a proof proving the corresponding mathematical
proposition \cite{martin_lof,ml_type_theory}. To illustrate this, consider the
examples in Figure~\ref{ml_type_theory_examples}. 
\footnote{These examples were provided by Prof. Richard Eisenberg in a
discussion.}. 

\begin{table}[h]
    \begin{tabular}{|c|c|c|c|}
        Haskell Type Signature & Math Proposition & Haskell type signature inhabited? & Proof exists? \\
        $\forall a. a \rightarrow a$ & $p \rightarrow p$ & True & True \\
        $\forall ab. (a,b) \rightarrow a$ & $ (p \wedge q) \rightarrow p$ & True & True \\
        $\forall ab. a \rightarrow b$ & $p\rightarrow q$ & False & False \\
        $\forall ab. a \rightarrow (a,b)$ & $p \rightarrow (p\wedge q)$ & False & False 
    \end{tabular}
    \caption{Comparison between Haskell type signatures and mathematical proofs to illustrate Curry-Howard Isomorphism}
    \label{ml_type_theory_examples}
\end{table}

From the figure, we see examples of the Curry-Howard isomorphism that forms the
fundamentals of Martin-L\"{o}f's type theory. We see that in all four cases, if
there is a function that inhabits the type signature, the mathematical
proposition presented is valid and if a function does not exist, the proposition
is invalid. Proving the isomorphism is beyond the scope of this literature
review but these are examples that should demonstrate that such an isomorphism
exists such that a strong relationship between proving mathematical propositions
and programming languages exists.

Mathematicians took an interest in creating a programming language based on the
Curry-Howard isomorphism and Martin-L\"{o}f type theory, since the Curry-Howard
isomorphism meant that a valid function that inhabits a type signature could be
equivalent to a proof. A dependently typed programming language based on
foundations in Martin-L\"{o}f Type Theory called NuPrl was first released in
1984 \cite{nuprl}. NuPrl is used as a \textit{proof assistant} that helps
mathematicians and programmers formalize proofs \cite{nuprl}. Dependently typed
proof assistants like NuPrl found a home at the intersection between programming
language enthusiasts interested in total program correctness and constructive
mathematicians interested in systems where mathematical formalisms could be
systematically encoded. Other proof assistants with support for dependent types
followed: Coq (1989) \cite{coquand1988calculus}, ALF (1990) \cite{alf}, Agda
(1999) \cite{norell:thesis}.

While there are now robust theorem provers that incorporate dependent types,
recent work has emerged concerned with bringing them into mainstream programming
and software development. Idris (2011) was designed with general purpose
programming in mind \cite{tdd_book}. F* (2011) was introduced by Microsoft as a
dependently typed language specifically designed around solving problems in
secure distributed programming \cite{fstar_distributed_programming}. In addition
to the development of new programming languages with dependent type systems
built into the language by design, work exists to mainstream dependent types
into more prominent programming languages. The most active and promising
mainstreaming work is on Haskell \cite{eisenberg2016,gundry2013}. 