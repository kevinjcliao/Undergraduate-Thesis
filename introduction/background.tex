\section{Background}
In this thesis, I will explore existing literature around practical real-world
applications of dependent types. I'll look at five examples to illustrate these
real-world applications. They range in domain from relational databases to
low-level bit manipulation. The hope is to demonstrate that dependent types,
long confined to theoretical mathematics, have tremendous promise in helping
programmers build reliable and safe programs. After a review of existing
literature, I will then review the Single-Transferrable Vote (STV), a
preferential voting algorithm for electing multiple candidates to multiple seats
based on the ranked choices of the constituency. Finally, I will introduce
Palpatine, a STV Vote Counter with verification of cardinality built in
Idris, a dependently-typed programming language. 

A dependently typed programming language can have functions with types that
depend on a value. A function, at its core, is a map from a domain to a
co-domain. In other words, we expect there to be a certain set of elements in
the universe for which our function can give us a corresponding output. A way to
remove certain bugs in programs is to ensure that a function in a program is
indeed mapping from the correct set of potential inputs to the set of potential
outputs. 

One can consider static type systems as a way to narrow down the set of
potential inputs to the set of possible outputs. For example, a function that
takes in a string and outputs an integer gives certain compile-time guarantees
to its programmer. If compilation succeeds, the domain of this function will be
strictly limited to an element in the set of all possible strings in the
universe and the output will be limited to an element of the set of all possible
integers. 

However, consider, for example, a function \texttt{replicate} that takes in an
integer of value \texttt{b} and some element of type \texttt{a}. It then
\textit{replicates} the element \texttt{b} times, producing a length-indexed
vector of length \texttt{b} containing copies of the element of type \texttt{a}.
The Haskell type signature\footnote{This literature review assumes prior
knowledge with Haskell/Idris-style type signatures, since all the examples I've
provided are in this style. An introduction is available in the excellent
\textit{Real World Haskell}, available here:
http://book.realworldhaskell.org/read/types-and-functions.html} for such a
function could look like the one provided in Figure~\ref{replicate_dec}.

\begin{figure}[ht!!!!!!!]
    \caption{Non-dependently typed type signature for replicate}
    \label{replicate_dec}
    \begin{lstlisting}
        replicate :: Int -> a -> [a]
    \end{lstlisting}
\end{figure} 


Let's imagine that we have a list data type signature that contains information
not only about the type of the elements that the list contains, but also about
the length of the list. That is to say, the type signature of a length-indexed
vector can be expressed as Figure~\ref{vect}.

\begin{figure}[ht!!!]
    \caption{Using a length-indexed vector (Vect) data type.}
    \label{vect}
    \begin{lstlisting}
        Vect :: Int -> Type -> Type 
        -- A length-indexed vector has an integer denoting length, 
        -- and the type of its elements. 
        [1,2,3] :: Vect 3 Int
    \end{lstlisting}
\end{figure} 

% Replicate. 

Now that we've introduced the length of the \texttt{vect} type as part of its
type signature, we can write a much tighter type signature for our
\texttt{replicate} function. Essentially, the \texttt{replicate} function will
take in an integer with some value \texttt{len}, some element \texttt{x} of type
\texttt{elem}, and the function will produce some vector of length \texttt{len}
holding elements of type \texttt{elem}. This type signature is shown in
Figure~\ref{dtvect}. 

\begin{figure}
    \caption{Dependently typed type signature for replicate}
    \label{dtvect}
    \begin{lstlisting}
        replicate : (len : Int) -> (x : elem) -> Vect len elem
    \end{lstlisting}
\end{figure} 

What's peculiar about this is that the co-domain of this function is not
particularly fixed. In fact, it depends on the value of its input. For example,
if an integer of value 3 and some boolean is inputted, the co-domain of our
function is the set of all length-indexed vectors with length 3 and type
boolean. This is an example application of dependent types. What we've done is
created a function where the co-domain varies as the input value varies. 

The goal of dependent types is to write programs with extreme guarantees of
compile-time safety. We can use the types of the parameters of a function to
place tighter limits on the set of its co-domain, with the co-domain varying
depending on the values of the input parameters. 