
\chapter{Conclusion}

\section{Proposal for Future Work}

While I'm still uncertain about the direction to proceed, I'm interested in
looking at elections and e-voting and whether or not we can provide guarantees
of correctness to vote counting software written in dependently typed languages.
I take a particular interest in the Australian Senate voting verification
process because verification of vote count is an NP-complete problem
\cite{aus_senate}. 

Currently, the Australian government uses proprietary code to count Australian
senate ballots and has refused to release the source code after a Freedom of
Information Act request \cite{aus_senate_news}. If an open-sourced, verifiably
correct counting program were devised, we could greatly protect the integrity of
Australian elections. 

\section{Conclusion}
While many literature reviews begin by looking examining a problem and looking
for existing solutions, this literature review takes an opposite approach. The
broader problem we are trying to answer is one that crosses various domains and
engineering fields. To put it quite simply, \textit{programs crash}. Dependent
typed languages, long a toy for theoretical computer scientists and
constructivist mathematicians, are increasingly becoming realistic tools to
write code with necessary guarantees of correctness. In other words, dependent
types are a solution in search of a problem. 

In this literature review, I offered a brief summary as to what dependent types
are and what languages exist where dependent type functionality is available. I
then moved on to describe different applications of dependently typed
programming that exist in literature. I started by looking at Cryptol, a DSL for
cryptography, and showed how dependent types allow for implementing complex
pattern-matching that the language requires \cite{power_of_pi}. I then moved on
to discuss embedded data description languages, showing how one can describe how
data is structured and generate a parser out of such a description
\cite{power_of_pi}. I also examined the potential of dependent types to build a
typesafe database, eliminating runtime typechecking and thus reducing error and
increasing performance \cite{power_of_pi,eisenberg2016}. 

Outside of domain specific languages, I also showed the application of dependent
types to systems programming \cite{idris_systems_programming}, building
distributed systems \cite{fstar_distributed_programming} and units of
measurement \cite{gundry2013}. In this wide-ranging review, I've demonstrated
that as dependent types become brought into the mainstream, they have the
potential to empower programmers to build safe, robust programs in ways that
have not been possible before. 
