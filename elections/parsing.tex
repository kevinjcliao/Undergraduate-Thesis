\section{Parsing}

Having now discussed how Palpatine calculates the results of an STV election, we
now move to parsing data files to build the values required as parameters for an
STV election. Our file format is a variant of the one defined in
\cite{stv_haskell}. The file format consists of a list of the candidates on the
first line, followed by a colon and a number denoting the number of seats to
elect. A ballot is a list of preferences with the leftmost preference
representing the most desired candidate. The remainder of our file after our
first line consists of a list of ballot preferences with each line denoting a
separate ballot. Our ballots from our earlier example in this data format is
shown in Figure~\ref{stv_format}

\begin{figure}[ht!!!!!!!]
    \caption{Data format}
    \label{stv_format}
    \begin{lstlisting}
        [A,B,C]:2
        [A,C]
        [A,B,C]
        [A,C,B]
        [B,A]
        [C,B,A]
    \end{lstlisting}
\end{figure}

In order to read this file format, we must first read the first line, building
the total length-indexed vector of the number of candidates remaining and the
number of seats remaining. We then read each ballot, building a list of ballots
and indexing each according to the length-indexed vector of candidates
remaining. We'll go through reading the first line as there is interesting
information around verification and types present, but we assume that parsing
the rest of the ballots would be something easily accomplished by a functional
programmer. 

\subsection{Reading Line 1}

\begin{figure}[ht!!!!!!!]
    \caption{Data format}
    \label{read_first_line_code}
    \begin{lstlisting}
        ExVect : Type -> Type
        ExVect t = (n ** Vect n t)

        total
        readFirstLine : String -> Maybe (ExVect Candidate, Nat)
        readFirstLine input = do
            let lines = splitToLines input
            firstLine <- head' lines
            let splitted =split (== ':') firstLine
            strCand <- head' splitted
            strNum <- last' splitted
            listCand <- parseList strCand
            let cands = map (\x => MkCandidate x 0) listCand
            seats <- parsePositive strNum
            pure $ ((toVec cands), cast seats)
    \end{lstlisting}
\end{figure}

The code for \texttt{readFirstLine} is given in
Figure~\ref{read_first_line_code}. The \texttt{readFirstLine} function takes in
a string which contains the contents of our data file and it outputs two items
contained by the first line. The first is an \texttt{ExVect} containing all the
candidates possible, and the second is a natural number that denotes the number
of seats available to elect. 

Something important to note here is the \texttt{ExVect} data type. We discussed
the concept of \textit{Dependent Pairs} earlier on in the literature review
section of this thesis in \autoref{pbm_review}. The \texttt{ExVect} is a
dependent pair between a natural number on the left and a length-indexed vector
on the right. That is, the vector on the right must have the length $n$, the
same length as the value of the natural number on the left. If we don't know the
length of a vector in advance, the existential vector, or \texttt{ExVect} is a
way to create a vector wrapped inside of a dependent pair. We can pattern match
on the an \texttt{ExVect} to remove the length-indexed vector wrapped inside of
it later. \texttt{ExVect}s are isomorphic to lists. 

All of these operations take place in the syntactic sugar of \textit{do}
notation afforded to us by the \texttt{maybe} monad. That is, if a single one of
these operations fails, our whole function will return \texttt{Nothing}. 

\section{Running Palpatine}

\begin{figure}[ht!!!!!!!]
    \caption{Data format}
    \label{main_code}
    \begin{lstlisting}
        total
        runElection : String -> IO ()
        runElection fileName = do
            Right str <- readFile fileName
            | Left err => putStrLn "ERROR: ReadFile Failed."
            case readFirstLine str of
                Just ((p ** cands), seats) => do 
                    let ballots = readBallots str cands
                    let dq = (droopQuota (length ballots) (cast seats))
                    case stv
                        ( makeElection 
                          dq
                          seats
                          ballots
                          cands
                          emptyResults
                        ) of
                        e@(_,_,_,_,results) => do
                            putStrLn "Done running the election. The results are:"
                            putStrLn $ show results
        Nothing => putStrLn "Parse error."
    \end{lstlisting}
\end{figure}

Palpatine takes in a filename and calls the \texttt{readFile} function in the
Idris standard library, taking in a string containing the file name and
producing an \texttt{IO String}. We then want the candidates and the ballots
from the first line. To produce this, we call our \texttt{readLine} function
which produces an ExVect of candidates, and a natural number denoting the number
of seats encapsulated inside of the Maybe monad. If parsing line 1 does not
succeed, producing a value \texttt{Nothing}, we simply error. If it does, we
pattern match on the ExVect that is produced, extracting the length-indexed
vector that is produced and the natural number of seats. We then run
\texttt{readBallots} with the length-indexed vector we just extracted from the
\texttt{ExVect}. We now have enough information to invoke our STV counter. The
code for \texttt{runElection} is given in Figure ~\ref{main_code}. 