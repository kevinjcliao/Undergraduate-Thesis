\section{Parsing}

At its core, the goal of \textit{Palpatine} is to take in a list of ballots.
Each ballot denotes the voter's list of preferences. We also should know a list
of all candidates because that cannot be derived from the list of preferences.
There is a scenario where a candidate gets not a single preference from not a
single voter. To express this, I have a file format, that is a modified version
of the one defined in \cite{stv_haskell}. Currently, their work requires that
candidates be declared in a Haskell GADT. One can imagine that the PBM Data
Description Language described in chapter 2.2 of the literature review could
parse and build such a data type, however I find that we can simply include the
total list of candidates at the top of the file. 

The file format consists of a list of the candidates (in parentheses to denote
that it is ont a ballot). A ballot is a list of preferences with the leftmost
preference representing the most desired candidate. Our ballots from our earlier
example in this data format is shown in Figure~\ref{stv_format}

\begin{figure}[ht!!!!!!!]
    \caption{Data format}
    \label{stv_format}
    \begin{lstlisting}
        (A,B,C):2
        [A,C]
        [A,B,C]
        [A,C,B]
        [B,A]
        [C,B,A]
    \end{lstlisting}
\end{figure}

We will be storing our total list of candidates in a length-indexed vector,
where candidate corresponds to the string name. We want to parse our ballots
such that as we elect a candidate, we remove preferences for that candidate from
all of our ballots. The best way to do that is to have ballots be a list of
\textit{Finite Numbers}, \texttt{Fin}. The type \texttt{Fin n} refers to a
natural number that is in the set of all natural numbers between \texttt{Zero}
and $n$. Each preference of type \texttt{Fin n} in our ballot corresponds to an
index in our \texttt{candidates} vector. As we elect a candidate and remove them
from our \texttt{candidates} vector, we reindex all of the ballots, changing the
\texttt{Fin}s to reflect the new indices of our now smaller vector of
candidates. How this high-level description looks like in terms of data types is
shown in Figure~\ref{stv_data_types}.

\begin{figure}[ht!!!!!!]
    \caption{Data types for \textit{Palpatine}}
    \label{stv_data_types}
    \begin{lstlisting}
        Candidate : Type
        Candidate = String

        Candidates : Nat -> Type
        Candidates n = Vect n Candidate
        
        VoteValue : Type
        VoteValue = Double

        -- New ballot type has a list of fins and
        -- a double as a pair. 
        total
        Ballot : Nat -> Type
        Ballot n = (List (Fin n), VoteValue)
    \end{lstlisting}
\end{figure}

We can parse a vector by using a dependent pair. We will not be able to derive
the type of the vector from elsewhere in our program given that it is data that
is derived through I/O. Thus, if we had our input file as a string, a
\texttt{getCandidates} function would have the following type signature:
\texttt{getCandidates : String -> (n ** Candidates n)}, where $n$ refers to the
number of candidates. We can then take the \texttt{Candidates n} produced and
parse for the ballots. The overall function \texttt{parseBallots} has the type
signature: \texttt{parseBallots : String -> Candidates n -> List \$ Ballot n}.

Thus, we generate a \texttt{List \$ Ballot n} containing the ballots which
reflect user preferences and a \texttt{Candidates n} which is a length-indexed
vector of candidates. 