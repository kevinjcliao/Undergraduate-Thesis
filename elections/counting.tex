\section{Counting}

\textit{This section is currently rough. It is an outline of existing and
planned work. It is mostly based on an outline of the STV counter provided in}
\cite{stv_haskell} and code that I've implemented. 


We store our vote count as a hashmap with Strings as Keys representing each one
of our candidates and \texttt{VoteValue} (Doubles) as the values. In other
words, we define a type: 

\texttt{VoteCount : Type} \\
\texttt{VoteCount = SortedMap Candidate VoteValue}

After we have a list of ballots, we will want to distribute ballots between each
candidate. As each candidate is eliminated, we redistribute their ballots. In
other words, we should probably maintain another data structure, a Map of
Candidates to a list of ballots that correspond to the votes for that candidate.

The type for this data structure looks like as follows: 

\texttt{BallotDist : Nat -> Type} \\
\texttt{BallotDist n = SortedMap Candidate (List \$ Ballot n)}

All in all, it suffices to say that at every point in time, we will maintain a
list of elected candidates, a VoteCount map, a length-indexed vector of the
number of $n$ candidates left, and the number of $s$ seats to be filled. So, we
can define a constructor for the overall state of our STV election called the
\texttt{Election}. \texttt{Election} takes in two type variables, $n$ for the
candidates left and $s$ for the number of seats still available. The code for
such a type looks like this: 

\texttt{-- A tuple that stores: Number of candidates, list of ballots, seats available.}
\texttt{Election : Nat -> Nat -> Type}
\texttt{Election n s = (Candidates n, BallotDist n, s, VoteCount)}

We start by defining a function to calculate the Droop Quota and the surplus
transfer. \cite{stv_haskell} separates the vote counting process for the STV
system into the following formal steps: 

\subsection{Start}
We start by calculating the Droop Quota with a function \texttt{droopQuota : Nat
-> Nat -> Nat}. We then instantiate our VoteCount with a function
\texttt{instantiateVc : Candidates n -> VoteCount}, that simply instantiates a
VoteCount Map with all votecounts set at zero. As a start, we execute the
\texttt{count} step defined below for all the ballots. 

\subsection{Count}

\texttt{Count} takes the the \texttt{List \$ Ballot n} and splits them up by
first preference candidates. We see each ballot's first preferences and divide
the ballots into different piles, creating a \texttt{BallotDist n}. We then
build our \texttt{Election n s}, since we now have the candidates, number of
seats, the list of ballots and the votecount all instantiated.  As a
precondition, we check to make sure that neither $n$ and $s$ are not zero, and
that $n$ is greater than $s$, since if there are less candidates than the number
of seats available, all candidates are by default, elected. 

\begin{figure}[ht!!!!!!!]
    \caption{Count function}
    \label{firstcount}
    \begin{lstlisting}
        ||| firstCount runs through the ballots for the first time. It then
        ||| inserts the value into the VoteCount SortedMap. 
        ||| addVote is sort of an insertion function into the VoteCount sortedMap
        ||| addVote : Fin n -> Candidates n -> VoteValue -> VoteCount -> VoteCount
        total
        firstCount : Candidates n -> List (Ballot n) -> VoteCount -> VoteCount
        firstCount cands [] vc = vc
        firstCount cands (bal :: rest) vc = 
            firstCount cands rest newVc where
            newVc : VoteCount
            newVc = case bal of
                ((cand :: _), val) => addVote cand cands val vc
                ([], _)            => vc
    \end{lstlisting}
\end{figure}


\subsection{Elect}

After a \texttt{count} takes place, we now reach the \texttt{Elect} stage. The
Elect stage iterates through the VoteCount and checks to see which candidates
have now hit the Droop Quota. Candidates that have hit the quota are outputted
in a List of Candidates. I'm still uncertain how we would develop a type
signature for this, as we are unsure how many candidates could be eliminated and
elected in any given count so we don't know to what extent we will be
deprecating the natural numbers in the type of \texttt{Election n s}. I do know
that some \texttt{Election} data type will be outputted and a list of
candidates. The candidates will be removed from the candidate vector in the
Election and they will be added to the list of elected candidates. 

\begin{figure}[ht!!!!!!!]
    \caption{elect candidates}
    \label{electcands}
    \begin{lstlisting}
        total
        getElectedCands : Candidates n -> VoteCount -> Int -> (p ** Candidates p)
        getElectedCands cands vc dq = filter isOverQuota cands where
            isOverQuota : Candidate -> Bool
            isOverQuota cand = case getVoteVal cand vc of 
                Just voteVal => voteVal >= cast dq
                Nothing => False
    \end{lstlisting}
\end{figure}

\subsection{Transfer}
After we declare a winner, we transfer surplus ballots. This involves
calculating the surplus transfer value and modifying all of the ballots to no
longer preference the candidates that were just eliminated by mapping to the new
candidate vector that's inputted. We then move them to the new appropriate
BallotDist key for their next preference. 

\subsection{Elim}
\texttt{elim : Election (S n) s -> Election n s}. We eliminate a candidate when
we've exhausted the possible preferences. We run through the VoteCount,
determine which candidate has the least amount of votes and we go through our
BallotDist and redistribute their ballots. We then remove them from the
length-indexed vector of candidates. 

\subsection{Complete}
The base case for our algorithm is when there are no seats left to be elected,
or there are no candidates left to be elected. The \texttt{complete} function
has the type signature \texttt{Election n s -> List Candidate}. Essentially, we
pattern match on the number of seats left and the number of available
candidates. If either are zero, we have hit a base case, and we simply return
the list of elected candidates. 

By running \texttt{count}, \texttt{elect}, \texttt{transfer} continuously until
we can no longer do so, and then running \texttt{elim} before iterating through
\texttt{count}, \texttt{elect}, \texttt{transfer}. I assert that this algorithm
must terminate. How such a termination will be proven will be discussed in the
next section.

\subsection{Final Algorithm for STV}

\begin{figure}[ht!!!!!!!]
    \caption{Final STV type signature. }
    \label{stv_idris}
    \begin{lstlisting}
        total
        ||| Running an STV election involves taking in the candidates, the seats, the
        ||| ballots and producing a list of candidates to take that seat. 
        stv : Candidates n 
            -> List $ Ballots n 
            -> (seats : Nat) 
            -> (Candidates seats, Candidates, n - seats)
    \end{lstlisting}
\end{figure}

The final code for STV guarantees means we are guaranteed that upon termination,
we will elect the correct number of seats to the senate and the sum of the
eliminated candidates and the elected candidates will equal the number of input
candidates. 

List of potential improvements(????)
\begin{enumerate}
    \item Type level guarantees of correctness from the Droop Quota and Surplus Quota functions?
    \item Using type level rationals instead?
    \item Could we enforce that these two lists of candidates are disjoint at the type level?
\end{enumerate}