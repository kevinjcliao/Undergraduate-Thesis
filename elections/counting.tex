\section{Counting}

\subsection{Functional Paradigm Pseudocode for the STV Algorithm}

The pseudocode given in the previous section describes, in an imperative sense,
what implementing the single transferrable vote would be like. For the purposes
of implementing STV in a dependently typed functional programming language, it
is necessary for us to translate such an algorithm into the functional style,
substituting the imperative loops for recursion and avoiding mutability. The
rough pseudocode for this is given in Algorithm~\ref{functional_stv}.

\begin{algorithm}
	\SetKwFunction{stv}{stv}
	\SetKwProg{Fn}{Function}{}{}
	\Fn{\stv{ballots, seats, candidates, judged}}{
		\eIf{candidates is not empty}{
			\eIf{highestCandidate >= droopQuota}{
				let next := elect(highestCandidate)\\
				ballots = transferDown(ballots, highestCandidate)
			}{
				let next := elim(lowestCandidate)
				ballots = transferUp(ballots, lowestCandidate)
			}
			judged.append(next)\\
			candidates.remove(next)\\
			stv(ballots, seats, candidates, judged)
		}{
			judged
		}
    }
    \caption{Functional paradigm pseudocode for Palpatine.}
    \label{functional_stv}
\end{algorithm}

\subsection{High-Level Overview of STV Code} \label{highlevelstv}

To implement STV, we choose to iterate through a list of candidates, and make
judgments on each candidate. In other words, rather than build up a list of
candidates, we simply iterate through our candidates, assigning them a judgment
of either elected or unelected. Our overall code (ignore the types for the time
being), ends up looking like in Figure~\ref{overall_code}. Just simply think of
\texttt{Election} as a GADT where its first natural number paramater refers to
the number of remaining candidates left and the second natural number refers to
the number of candidates that have been judged.

In essence, our \texttt{stv} function repeatedly iterates over our
\texttt{Election} GADT, calling \texttt{processOne} repeatedly.
\texttt{processOne} checks to see if the highest scoring candidate is above the
droopQuota. If the candidate is above the droopQuota, we elect the candidate. If
not, we eliminate a candidate. If there are no seats left, as shown in the
second parameter of the first pattern match, we run \texttt{notElectedHead},
which just eliminates one candidate. If there are no seats left,
\texttt{notElectedHead} is called repeatedly until we eliminate all the
candidates.For the time being, we can ignore the \texttt{rewrite} function
present in the last line of the code. 

\begin{figure}[htbp!!!!!!!!]
	\caption{Ballots with initial score.}
	\label{overall_code}
	\begin{lstlisting}
        total
        processOne : Election (S r) j -> Election r (S j)
        processOne election@(_, Z, _, _, _)          = notElectedHead election
        processOne election@(dq, (S n), _, _, _) = case count election of
            counted@(_, _, _, cands, _) => if weCanElect dq cands
                then electOne counted
                else elimOne counted

        ||| Running an STV election involves taking in the candidates, the seats, the
        ||| ballots and producing a list of candidates to take the seats. 
        ||| Returns a tuple of elected candidates and unelected candidates.
        total
        stv : Election r j -> Election Z (r + j)
        stv         e@(_, _, _, Nil, _)      = e
        stv {r=(S n)} {j} e@(_, _, _, (_ :: _), _) = 
            rewrite plusSuccRightSucc n j in stv $ processOne e
    \end{lstlisting}
\end{figure}

\subsection{Data Structures}

We essentially need to store several pieces of information. This section
describes the data structures we will be using in our STV vote counter.

\begin{enumerate}
	\item We need to store the candidates remaining and also the score of each
	      candidate.
	\item We need to store a series of ballots. As we revalue ballots, we need
	      to revalue the ballots based on what each one of the ballots has voted for
	      when a given candidate is elected. In other words, each ballot contains
	      unexhausted preferences for candidates, candidates that the ballot has voted
	      for, and the value of the ballot.
	\item We need to store the judgments. In other words, we need to store a
	      growing list of each candidate that contains the judgment that the candidate
	      was either elected or not elected.
	\item We need to store the seats remaining. This is stored as a natural
	number. 
\end{enumerate}

\subsubsection{Candidates Data Type}

We choose to represent each candidate as a record type\footnote{Record Types are
	present in Haskell so discussing them is assumed as background knowledge. As
	always, Real World Haskell serves as a good primer for Record Types. }. Our
declaration for a record type contains its \texttt{CandidateName}, a type alias
for \texttt{String}, and its \texttt{VoteValue}, a type alias for
\texttt{Double}. The candidates who have not been judged yet are represented by
\texttt{Candidates}, a length-indexed vector with elements of the type
\texttt{Candidate}. In addition, we provide several accessors for our
\texttt{Candidates} type that will become useful later. While we have generic
getters and setters as well as incrementors and decrementors for the VoteValue
of a candidate, we also have a function \texttt{getHighestIndex} that takes in
an element of type \texttt{Candidates (S n)} and returns the index of the
candidate with the highest score. This index is represented by a number of type
\texttt{Fin (S n)}. \texttt{Fin (S n)} is a type that represents the set of
numbers between zero and the natural number \texttt{S n}. In other words, this
type signature guarantees to us that the index we return will be within bounds
of our length-indexed vector. We'll be using a lot more of \texttt{Fin}s
elsewhere in Palpatine.

As an aside here, unlike Haskell, Idris does not support the automatic
\textit{deriving} keyword which allows a programmer to easily extend a type to
support type classes. I had to define the implementations for \texttt{Eq, Ord,
	Show} for \texttt{Candidate} and all subsequent custom data types by hand. All
the code relevant to candidates is shown in Figure~\ref{candidates_code}

\begin{figure}[htbp!!!!!!!!!!!!!!]
	\caption{Candidates data type with relevant accessors and setters}
	\label{candidates_code}
	\begin{lstlisting}
        CandidateName : Type
        CandidateName = String

        VoteValue : Type
        VoteValue = Double

        record Candidate where
            constructor MkCandidate
            candName : CandidateName
            candValue : VoteValue
        
        Candidates : Nat -> Type
        Candidates n = Vect n Candidate

        ||| Returns the index and score of the candidate with the highest score. 
        getHighestIndex : Candidates (S n) -> (Fin (S n), VoteValue)

        ||| Returns the index and score of the candidate with the lowest score. 
        getLowestIndex : Candidates (S n) -> (Fin (S n), VoteValue)

        ||| Adds a votevalue to the score of a candidate. 
        addVoteVal : Fin n -> Candidates n -> VoteValue -> Candidates n
        
        ||| Decrements a votevalue from the score of a candidate. 
        decVoteVal : Fin n -> Candidates n -> VoteValue -> Candidates n

        ||| Removes a candidate from the vector of candidates. 
        removeCand : Fin (S n) -> Candidates (S n) -> Candidates n

        ||| Makes a new candidate with score 0. 
        makeNewCandidate : CandidateName -> Candidate
    \end{lstlisting}
\end{figure}

\subsubsection{Storing results}

Like \texttt{Candidate}, we also choose to represent each judgment as a record
type. A judgment consists of a candidate name and whether or not that candidate
was elected. To represent that, we must declare a new Algebraic Data Type called
\texttt{Judgment}. \texttt{Judgment} simply contains two values and is in a way
analogous to \texttt{Bool}. It simply tells us if a \texttt{candidate} is
\texttt{Elected} or \texttt{NotElected}. We then have a \texttt{Results} data
type that is a length-indexed vector containing elements of type
\texttt{Judgment}. Like \texttt{Candidate} and \texttt{Candidates}, I also had
to write implementations for the various type classes. It was especially
frustrating for \texttt{Judgment}, which I felt was a very generic ADT that
Haskell's \texttt{deriving} keyword would've handled just fine. The code is
given in Figure~\ref{judgments_code}

The \texttt{Judged} data type is a record type that contains an element of type
\texttt{CandidateName} (recall this is a type alias for a string) and an element
of type \texttt{Judgment}. \texttt{Results} of our STV counter are stored in a
length-indexed vector containing elements of type \texttt{Judged}.

\begin{figure}[htbp!!!!!!!!!!!!!!]
	\caption{Results data type with relevant accessors and setters}
	\label{judgments_code}
	\begin{lstlisting}
        data Judgment = Elected | NotElected

        Results : Nat -> Type
        Results n = Vect n Judged
        
        emptyResults : Results Z
        emptyResults = Nil

        elect : Candidate -> Judged
        elect cand = MkJudgment (candName cand) Elected

        dontElect : Candidate -> Judged
        dontElect cand = MkJudgment (candName cand) NotElected
    \end{lstlisting}
\end{figure}

\subsubsection{Storing Ballots}


\begin{figure}[htbp!!!!!!!!!!!!!!]
	\caption{Ballots data type with relevant accessors and setters}
	\label{ballots_code}
    \begin{lstlisting}
        Preferences : Nat -> Type
            Preferences n = List $ Fin n

        record Ballot (n : Nat) where
            constructor MkBallot
            votedFor : List CandidateName
            prefs : Preferences n
            value : VoteValue
        
        Ballots : Nat -> Type
        Ballots n = List $ Ballot n

        total
        nextCand : Ballot n -> Maybe $ Fin n
        nextCand ballot         = case getPrefs ballot of
            [] => Nothing
            (cand :: _) => Just cand

        total
        restCand : Ballot n -> Ballot n
        restCand ballot         = case getPrefs ballot of
            []          => ballot
            (_ :: rest) => replacePrefs rest ballot

        ballotDidElectCandidate : CandidateName -> Ballot n -> Bool
        ballotDidElectCandidate name ballot = elem name (votedFor ballot)
        
        addToVotedFor : Candidates n -> Ballot n -> Ballot n
        addToVotedFor cands ballot = case nextCand ballot of
            Just topPrefIndex => let name = candName (index topPrefIndex cands) in 
                record { 
                    votedFor = name :: (votedFor ballot) 
                } (restCand ballot)
            Nothing => ballot
        
        total
        changeBallotIfIsCand : CandidateName -> VoteValue -> Ballot n -> Ballot n
        changeBallotIfIsCand cand vv ballot = if ballotDidElectCandidate cand ballot
            then replaceValue vv ballot
            else ballot
    \end{lstlisting}
\end{figure}


A ballot contains, as described above, three things. A list of unexhausted
preferences, representing candidates the ballot has yet to vote for, a value,
denoting the current score of the ballot, and a list of candidates that the
ballot has already voted for. This can be represented as a record type. The
record type \texttt{Ballot} contains a constructor, a list of candidates that
the ballot has voted for, and a list of \texttt{Preferences n} that reflects the
preferences of each ballot. It finally contains a \texttt{VoteValue}, or
\texttt{Double} that reflects the value of the ballot. \texttt{Preferences n} is
an algebraic data type that takes in a natural number representing the number of
candidates left in \texttt{Remaining}. Each Fin it contains is bounded by the
index of the candidates remaining. In other words, each Fin stored as a
\texttt{Preference} maps to an element in \texttt{Remaining}. The code for storing
and handling ballots is given in Figure~\ref{ballots_code}

We provide several functions to act on ballots. \texttt{nextCand} returns the
index of the ballot's next preference if one exists. \texttt{restCand} returns
the rest of the ballot's preferences. \texttt{ballotDidElectCandidate} takes a
\texttt{CandidateName} and checks to see if the ballot has voted for the
candidate. \texttt{addToVotedFor} adds the next candidate preferenced to the
list of candidates that this ballot has voted for. \texttt{changeBallotIfIsCand}
is what's used when changing a ballot's value. We check to see if our ballot has
voted for the candidate and if it has, we change its value to the new value
given. 

\subsubsection{The Election GADT}

Now we can group together all the different data structures we've built to make
the \texttt{Election} data type, which contains the Droop Quota as an
\texttt{Integer}, the number of seats available as a natural number, the
ballots, the candidates and the results. The code for the election GADT is given
in Figure~\ref{election_type_code}. 

\begin{figure}[ht!!!!!!!!!!!!!!]
	\caption{Election Type}
	\label{election_type_code}
	\begin{lstlisting}
        Election : Nat -> Nat -> Type
        Election r j = (Int, Nat, Ballots r, Candidates r, Results j)

        makeElection : Int 
                    -> Nat 
                    -> Ballots r
                    -> Candidates r 
                    -> Results j 
                    -> Election r j
        makeElection i n b r j = (i, n, b, r, j)
    \end{lstlisting}
\end{figure}

\newpage

\subsection{Application of Data Structures to STV}

Having established our data structures, we now move back to the high-level STV
code we introduced earlier in Section~\ref{highlevelstv}. The code in
Figure~\ref{overall_code} has been reproduced in Figure~\ref{overall_code2} for
the sake of convenience.

\begin{figure}[ht!!!!!!!!]
	\caption{Reproduced overall STV code.}
	\label{overall_code2}
	\begin{lstlisting}
        total
        processOne : Election (S r) j -> Election r (S j)
        processOne election@(_, Z, _, _, _)          = notElectedHead election
        processOne election@(dq, (S n), _, _, _) = case count election of
            counted@(_, _, _, cands, _) => if weCanElect dq cands
                then electOne counted
                else elimOne counted

        ||| Running an STV election involves taking in the candidates, the seats, the
        ||| ballots and producing a list of candidates to take the seats. 
        ||| Returns a tuple of elected candidates and unelected candidates.
        total
        stv : Election r j -> Election Z (r + j)
        stv         e@(_, _, _, Nil, _)      = e
        stv {r=(S n)} {j} e@(_, _, _, (_ :: _), _) = 
            rewrite plusSuccRightSucc n j in stv $ processOne e
    \end{lstlisting}
\end{figure}

Now, notice that we've defined we're doing type-level arithmetic on the natural
numbers in our Election data type. Our \texttt{stv} function takes in a value of
type \texttt{Election} with $r$ remaining candidates and $j$ made judgments. Our
function is guaranteed to return an element of type \texttt{Election} with no
remaining candidates $Z$ and judgments on all the candidates $r+j$. As mentioned
above, our \texttt{stv} function calls \texttt{processOne} repeatedly. Now we
also have guarantees that \texttt{processOne} takes in an \texttt{Election} with
at least one candidate remaining. Notice that the first natural number of
\texttt{Election} is \texttt{(S r)}, so it must be of one or greater since it is not
Zero, or $Z$. 

What's left then, is to introduce several functions that were mentioned in the
\texttt{stv} and \texttt{processOne} functions. We must first introduce the
function for counting the first preferences of each ballot. We must then
introduce the function for electing a candidate and eliminating a candidate. We
must then introduce the predicate \texttt{weCanElect}. We must also introduce
\texttt{notElectedHead} which just eliminates a candidate indiscriminately after
all the seats have been filled. Finally, we will comment on the use of the
\texttt{rewrite} function in the last line of \texttt{stv}. 

\subsubsection{Counting First Preferences}

\begin{figure}[ht!!!!!!!!]
	\caption{Code for counting the first preference of each ballot.}
	\label{counting_code}
    \begin{lstlisting}
        ||| Does a full count of the ballots by taking the top preference and the vote
        ||| value and putting it into the candidates votes for the given candidate. 
        total
        count : Election r j -> Election r j
        count {r} election@(dq, seats, ballots, cands, results) = 
            makeElection dq seats newBallots newCands results where
                countBallot : Ballot r -> Candidates r -> Candidates r
                countBallot ballot cands = case nextCand ballot of
                    Just topPrefIndex => 
                        addVoteVal topPrefIndex cands $ ballotValue ballot
                    Nothing           => cands
                countBallots : Ballots r -> Candidates r -> Candidates r
                countBallots Nil cands       = cands
                countBallots (x :: xs) cands = countBallots xs $ countBallot x cands
                newBallots : Ballots r
                newBallots = map (addToVotedFor cands) ballots
                newCands : Candidates r
                newCands = countBallots ballots cands
    \end{lstlisting}
\end{figure}

As mentioned above, we store the score that each candidate currently has in the
\texttt{Candidate} record type. Thus, counting a ballot takes the ballot's first
preference, removes it and adds it to the list of candidates that ballot has
voted for. The ballot value is then added to the score of the \texttt{Candidate}
inside of the length-indexed vector remaining. Our \texttt{countBallot} function
takes the score of each ballot and adds it to its first preference. Our
\texttt{newBallots} are created by mapping \texttt{addToVotedFor} over our list
of ballots, which eliminates the next preferenced candidate and moves it to the
list of candidates that the ballot has voted for. We then take our newBallots
and newCandidates and rebuild our election of type \texttt{Election r j}. 

\subsubsection{Removing candidates once all have been elected}

\begin{figure}[ht!!!!!!!!]
	\caption{Code for when we are just eliminating candidates indiscriminately.}
	\label{not_elected_head_code}
    \begin{lstlisting}
        ||| After a candidate was just eliminated from remaining, this iterates through
        ||| the ballots and reindexes them according to the new candidate indices rather
        ||| than the old one. 
        total
        reindexBallots : Ballots (S r) -> Candidates (S r) -> Candidates r -> Ballots r
        reindexBallots {r} ballots oldCands newCands = 
            map reindexBallot ballots where
                reindexCand : Fin $ S r -> Maybe $ Fin r
                reindexCand oldPref = findIndex (\x => candName x == cn) newCands where
                    cn : CandidateName
                    cn = candName $ index oldPref oldCands
                reindexBallot : Ballot (S r) -> Ballot r
                reindexBallot ballot = newBallot where
                    newPrefs : Preferences r
                    newPrefs = mapMaybe reindexCand $ getPrefs ballot
                    newBallot : Ballot r
                    newBallot = MkBallot (votedFor ballot) newPrefs (value ballot)

        total
        notElectedHead : Election (S r) j -> Election r (S j)
        notElectedHead election@(dq, seats, ballots, (x :: xs), results) = 
            makeElection 
                dq 
                seats 
                (reindexBallots ballots (x :: xs) xs)
                xs
                ((dontElect x) :: results)
    \end{lstlisting}
\end{figure}

Notice how in Figure~\ref{overall_code2} that when \texttt{processOne} sees that
there are no seats left, it calls a function called \texttt{notElectedHead}.
\texttt{notElectedHead} simply makes a judgment that the next candidate has
failed to be elected. The code for this function is provided in
Figure~\ref{not_elected_head_code}. It takes in the existing election and it does
three things: 

\begin{enumerate}
    \item It removes the head from the length-indexed vector of remaining candidates
    \item It adds the head with the judgment of not elected to the vector of judged candidates. 
    \item It reindexes the ballots, so that the preferences are now a list of
    indices that correspond to the new length-indexed vector of candidates with
    the head candidate removed. 
\end{enumerate}

Most of this code does not need to be explained since it just involves pattern
matching on and manipulating the head and rest of a length-indexed vector. What
deserves comment is the presence of the function \texttt{reindexBallots}.
\texttt{reindexBallots} will be used elsewhere in our election and elimination
functions as well. As described above, it takes in \texttt{Ballots}, each
indexed according to a previous, larger length-indexed vector of remaining
candidates. It then indexes them according to the next, smaller length-indexed
vector of candidates. This function does a \texttt{map} over the ballots,
calling the function \texttt{reindexBallot} on each function.
\texttt{reindexBallot} does a mapMaybe of \texttt{reindexCand}, grabbing each
preference from the ballot and seeing whether it is present in the new ballot.
\texttt{mapMaybe} is a \texttt{map} function that does not append to a list if
the result is \texttt{Nothing}. In other words, if the candidate being indexed
is not in the new length-indexed vector, it won't be part of the returned list
of preferences in the ballot. 

\subsubsection{Eliminating a candidate}

\begin{figure}[ht!!!!!!!!]
	\caption{Code for eliminating a candidate}
	\label{elim_cand_code}
    \begin{lstlisting}
        ||| elimOne eliminates a candidate. It chooses the lowest valued candidate
        ||| and then makes a judgment that the candidate is eliminated. It then
        ||| reindexes the ballots without changing the vote value of the ballots. 
        total
        elimOne : Election (S r) j -> Election r (S j)
        elimOne {r} {j} election@(dq, seats, ballots, cands, results) = 
            (dq, seats, newBallots, newCands, newResults) where
                lowestCandIndex : Fin $ S r
                lowestCandIndex = case getLowestIndex cands of
                    (i, _) => i
                result : Judged
                result = dontElect $ getCand lowestCandIndex cands
                newResults : Results (S j)
                newResults = (result :: results)
                newCands : Candidates r
                newCands = removeCand lowestCandIndex cands
                newBallots : Ballots r
                newBallots = reindexBallots ballots cands newCands
    \end{lstlisting}
\end{figure}

When we eliminate a candidate, the first thing we do is we get the index of the
candidate with the lowest score. We then make a judgment of this candidate as a
candidate we are not going to elect by calling our \texttt{dontElect} function
to create an element of type \texttt{Judged}. We then add this to our
length-indexed vector of judgments. Whereas previously this length-indexed
vector was of length $j$, it is now of length \texttt{(S j)}. We remove that
candidate from our length-indexed vector of candidates. Whereas previously our
length-indexed vector of remaining candidates was of length \texttt{(S r)},
having just eliminated the candidate with the lowest score, it now has length
\texttt{r}. We reindex the candidates according to our new length-indexed vector
of remaining candidates and we rebuild our election. This election now has type
\texttt{Election r (S j)}, indicating that we have successfully judged one
candidate in the process. 

\subsubsection{Electing a candidate}

\begin{figure}[ht!!!!!!!!]
	\caption{Code for electing a candidate.}
	\label{elect_cand_code}
    \begin{lstlisting}
        ||| electOne elects a candidate. It takes in the highest candidate index and
        ||| makes a judgment on that candidate as the new elected candidate. It then
        ||| redistributes the preferences according to the new transfer value. 
        total
        electOne : Election (S r) j -> Election r (S j)
        electOne {r} {j} election@(dq, seats, ballots, cands, results) = 
            (dq, seats, newBallots, newCands, newResults) where
                highestCandIndex : Fin $ S r
                highestCandIndex = case getHighestIndex cands of
                    (i, _) => i
                highestCand : CandidateName
                highestCand = candName $ index highestCandIndex cands
                highestCandValue : VoteValue
                highestCandValue = candValue $ index highestCandIndex cands
                result : Judged
                result = elect $ getCand highestCandIndex cands
                newResults : Results (S j)
                newResults = (result :: results)
                newCands : Candidates r
                newCands = removeCand highestCandIndex cands
                newBallotVal : VoteValue
                newBallotVal = transferValue dq highestCandValue
                -- This creates ballots with newBallotVal (if they preferenced highest
                -- cand first). 
                ballotsWithNewValue : Ballots (S r)
                ballotsWithNewValue = 
                    map (changeBallotIfIsCand highestCand newBallotVal) ballots
                newBallots : Ballots r
                newBallots = reindexBallots ballotsWithNewValue cands newCands
    \end{lstlisting}
\end{figure}

Electing a candidate works in similar ways to eliminating one. We start by
getting the highest index. We then get the candidate name and value. We then
build a judgment of type \texttt{Judged} that declares our candidate to be
elected. After appending our new judgment to the length-indexed vector of
judgments, we remove the highest candidate from the length-indexed vector of
remaining candidates. We then calculate the transfer value, and then run through
all the ballots, changing their value if they voted for the candidate. Finally,
we reindex the ballots according to the newly created, one size smaller
length-indexed vector of remaining candidates. We then rebuild our election, now
of type \texttt{Election r (S j)}

\subsubsection{Why call rewrite?}


\begin{figure}[ht!!!!!!!!]
	\caption{STV code without rewrite}
	\label{no_rewrite_code}
    \begin{lstlisting}
        total
        stv : Election r j -> Election Z (r + j)
        stv         e@(_, _, _, Nil, _) = e
        stv e@(_, _, _, (_ :: _), _)    = stv $ processOne e
    \end{lstlisting}
\end{figure}

\begin{figure}[ht!!!!!!!!]
	\caption{Error message produced by Idris compiler.}
	\label{idris_error_code}
    \begin{lstlisting}
        When checking right hand side of stv with expected type
                Election 0 (S len + j)

        Type mismatch between
                Election 0 (len + S j) (Type of stv (processOne e))
        and
                (Int,
                 Nat,
                 List (Ballot 0),
                 Vect 0 Candidate,
                 Vect (S (plus len j)) Judged) (Expected type)

        Specifically:
                Type mismatch between
                        plus len (S j)
                and
                        S (plus len j)
    \end{lstlisting}
\end{figure}

The last item to explain in the code of our STV function is why we call
\texttt{rewrite}. If we didn't call it, the code for \texttt{stv} would look
like the code given in Figure~\ref{no_rewrite_code}. The Idris compiler produces
an error, telling us that it is unable to compile. This error message is
reproduced in Figure~\ref{idris_error_code}.

What does this error message mean? It means that this function has an output of
\texttt{Election 0 (S len + j)}. However, the function makes a recursive call to
\texttt{processOne}, which produces an Election of type \texttt{Election len (S
j)}. The Idris compiler is unable to deduce that \texttt{(S len + j)} is the
same as \texttt{len + (S j)}. Luckily, there is a function in the Idris Prelude
that operates on Natural Numbers that proves this equality for us. It is called
\texttt{plusSuccRightSucc}. What we need to do is use this proof to rewrite some
of our type variables. \texttt{plusSuccRightSucc} has type signature:
\texttt{total plusSuccRightSucc : (left : Nat) -> (right : Nat) -> S (left +
right) = left + (S right)}. 

\begin{figure}[ht!!!!!!!!]
	\caption{STV code with the rewrite.}
	\label{with_rewrite_code}
	\begin{lstlisting}
        total
        stv : Election r j -> Election Z (r + j)
        stv         e@(_, _, _, Nil, _)      = e
        stv {r=(S n)} {j} e@(_, _, _, (_ :: _), _) = 
            rewrite plusSuccRightSucc n j in stv $ processOne e
    \end{lstlisting}
\end{figure}

Thus, to call this function to demonstrate the necessary equality, we need to
call \texttt{plusSuccRightSucc} with \texttt{len} and $j$, where \texttt{len}
refers to the natural number preceeding $r$ and $j$ refers to the number of
judgments in the input \texttt{Election}. Thus, we call \texttt{rewrite} to
allow the Idris compiler to recognize this equality and our code ends up looking
like in Figure~\ref{with_rewrite_code}. 

Thus, in this section, we have introduced all the code that Palpatine uses to
count STV ballots. 