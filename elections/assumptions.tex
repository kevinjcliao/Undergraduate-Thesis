\section{Simplifying Assumptions}

There are a lot of simplifying assumptions in this text. We operate under the
assumption that a list of ballots that describe the accurate preferences of the
voters can be found. This entails a lot of simplifying assumptions which
represent ongoing problems in voting. If paper ballots are employed, one must be
certain that the paper ballots are read into the data format correctly. This is
fraught with potential for interference at every step, although is possible.
Similarly, if we are using electronic ballots, we must be certain that the vote
recorded in our data files matches the vote input by the user. There is a wide
body of literature on these issues and they fall outside of the scope of this
thesis \cite{election_safety}. Work, for example, exists on using formal methods
to produce hardware-verifiable tamper-proof ballots with PROM
\cite{prom1,prom2}. 

I proceed with these assumptions in mind, that someone running a piece of ballot
counting software possesses a list of accurate preferences that are protected from
tampering end-to-end. I acknowledge that there are significant challenges to
guarantee each one of these simplifications, but they are outside the scope of
my thesis. 