\section{Simplifying Assumptions}

There are a lot of simplifying assumptions in this text. We operate under the
assumption that a list of ballots that accurately describe the honest
preferences of the voters can be found. This in-and-of-itself means a lot of
simplifying assumptions which represent ongoing problems in online voting. If
paper ballots are employed, one must be certain that the paper ballots are read
into the data format correctly. This is fraught with potential for interference
at every step, although is possible. Similarly, if we are using electronic
ballots, we must be certain that the vote recorded in our data files matches the
vote inputted by the user. There is a wide body of literature on these issues
and they fall outside of the scope of this thesis. Work, for example, exists on
using formal methods to produce hardware-verifiable tamper-proof ballots with
PROM \cite{prom1}, \cite{prom2}. 

The question of whether ballots are \textit{honest} is a class of interesting
problems. In a traditional first-past-the-post, single-choiced ballot voting
system, one might decide to vote for a majority-party candidate with a larger
chance of winning instead of a candidate with a lower chance of winning. Third
party candidates become `vote spoilers' and voters are forced to engage in
`strategic voting'. They must choose a candidate that not necessarily best
reflects their political beliefs, but reflects the pragmatic reality of the
electoral system that they are presented with. Now, having introduced the
ability to rank, the question of whether a ballot \textit{honesty} takes on a
deeper meaning. Could someone maliciously preference the ballot in such a way
that the voting algorithm would favor a dishonest candidate preference? (e.g. to
elect candidate A, it is better to preference B, A, C instead of A, B, C). It
turns out that calculating dishonest preferences to game an STV election is an
NP-Complete problem \cite{strategic}. Calculating the margin of votes require to
game an STV election is also NP-Complete \cite{margin_of_victory}. 

I proceed with these assumptions in mind, that someone running a piece of ballot
counting software possesses a list of honest preferences that are protected from
tampering end-to-end. I acknowledge that there are significant challenges to
guarantee each one of these simplifications, but they are outside the scope of
my thesis. 