\chapter{Relevant Background on the Single Transferrable Vote}

Having provided a grounding in dependently-typed programming, I now move to the
voting system that I take an interest in. I take an interest in the Single
Transferable Vote (STV) system that is used to count voter preferences in many
developed democracies. STV is used in Ireland, Australia, New Zealand, and lower
level elections across the United Kingdom and the United States. Many variants
of the system exist and I concern myself with the method of STV currently used
to count votes in the Australian Senate. 

\section{A sample STV election}
\textit{This is an application of a description of the Australian Senate STV Counting Process as described by the Australian Electoral Commision.}

In order to demonstrate how the STV system works, it would help to run a sample
election, provided below. This sample election data is not original and obtained
from prior literature for an STV vote counter in Haskell and Coq
\cite{stv_haskell}. Suppose we have three candidates, $[A, B, C]$, running for
two available seats. Now suppose that we have 5 voters who each have sent in
preferences to be counted. The preferences are shown in
Figure~\ref{sample_election}. Each voter's preferences are referenced by a list
where the leftmost element represents the most preferred candidate and the
rightmost candidate presents the voter's least preferred. There is no obligation
to number all candidates. 


\begin{figure}[ht!!!!!!!!]
    \caption{Ballots for a sample election to be run. }
    \label{sample_election}
    \begin{lstlisting}
        [A,C]
        [A,B,C]
        [A,C,B]
        [B,A]
        [C,B,A]
    \end{lstlisting}
\end{figure}

The first thing to calculate is the \textit{Droop Quota}. This represents the voting `score' a candidate must have before they are elected. The Droop Quota is calculated in the Australian Senate by a formula shown below. 

$$
S = floor(\frac{b}{n + 1}) + 1
$$

In this formula, $b$ refers to the number of ballots, $n$ refers to the number
of seats available, and the \textit{floor} operation refers to rounding the
result down. The Droop Quota of our sample election is, according to this formula, 

$$
S = floor(\frac{5}{2 + 1}) + 1 = floor(\frac{5}{3}) + 1 = 1 + 1 = 2
$$

Now, we assign a \textit{weight} to each ballot. Our ballots are now: 

\begin{figure}[ht!!!!!!!!]
    \caption{Ballots with initial score.}
    \label{sample_election2}
    \begin{lstlisting}
        ([A,C], 1)
        ([A,B,C], 1)
        ([A,C,B], 1)
        ([B,A], 1)
        ([C,B,A], 1)
    \end{lstlisting}
\end{figure}

We now start by counting everyone's first preferences. $A$ receives 3 ballots
each with the weight of 1.0. Thus, $A$ has reached the Droop Quota and $A$ is
elected. No one else has hit the Droop Quota yet. Now, we transfer the surplus.
We remove the ``head'' from all the ballots that preferenced A and we transfer
the surplus votes. The new value of the votes is calculated as follows: 

$$
v = s/n
$$

where $v$ is the surplus transfer value, $s$ is the surplus ballots and $n$ is
the total number of ballots that preferenced the candidate first. By applying
this formula, the Droop Quota is 2, there are 3 votes for $A$ and so there is a
surplus of 1. $B, C$ each received one vote. This means the surplus is expressed
by $\frac{1}{3}$. In other words, our ballots now look like: 

\begin{figure}[ht!!!!!!!!]
    \caption{Ballots with initial score.}
    \label{sample_election2}
    \begin{lstlisting}
        ([C], 1/3)
        ([B,C], 1/3)
        ([C,B], 1/3)
        ([B], 1)
        ([C,B], 1)

        elected: [A]

        scores: 
        (B, 1)
        (C, 1)
    \end{lstlisting}
\end{figure}

We then immediately start counting the surplus transfer from our elected candidates. In other words, C gets an increase in score of $\frac{2}{3}$ and $B$ gets an increase in score of $\frac{1}{3}$. The scores of these ballots remain the same. 

\begin{figure}[ht!!!!!!!!]
    \caption{Ballots with initial score.}
    \label{sample_election3}
    \begin{lstlisting}
        ([], 1/3)
        ([C], 1/3)
        ([B], 1/3)
        ([B], 1)
        ([C,B], 1)

        elected: [A]

        scores: 
        (B, 4/3)
        (C, 5/3)
    \end{lstlisting}
\end{figure}

Now, given we have exhausted the surplus of elected candidates and there is
still one vacancy available, we remove the candidate with the lowest score,
which is $B$. We look at the votes that went to candidate $B$. There is one
exhausted ballot $[]$, and another ballot: $[C]$ with transfer value $1/3$. We
transfer this ballot over to $C$. We now have one candidate left in the running
and one seat open. This means that $C$ is elected by virtue of being the only
candidate left. 

The result of our election is thus, $[A,C]$ being elected to the senate.  

\section{Why STV?}
\textit{Coming in final draft of the thesis.} \\
In short, this will be a section on how STV allows someone to genuinely express
their candidate preferences in confidence that their vote will not be ``thrown
away.''

\section{Benefits and drawbacks of STV in summary}
\textit{Coming in final draft of the thesis.}

\section{Simplifying Assumptions}

There are a lot of simplifying assumptions in this text. We operate under the
assumption that a list of ballots that describe the accurate preferences of the
voters can be found. This entails a lot of simplifying assumptions which
represent ongoing problems in voting. If paper ballots are employed, one must be
certain that the paper ballots are read into the data format correctly. This is
fraught with potential for interference at every step, although is possible.
Similarly, if we are using electronic ballots, we must be certain that the vote
recorded in our data files matches the vote input by the user. There is a wide
body of literature on these issues and they fall outside of the scope of this
thesis \cite{election_safety}. Work, for example, exists on using formal methods
to produce hardware-verifiable tamper-proof ballots with PROM
\cite{prom1,prom2}. 

I proceed with these assumptions in mind, that someone running a piece of ballot
counting software possesses a list of accurate preferences that are protected from
tampering end-to-end. I acknowledge that there are significant challenges to
guarantee each one of these simplifications, but they are outside the scope of
my thesis. 

\chapter{Introducing Palpatine: An STV Vote Counter in Idris.}
\section{Parsing}

At its core, the goal of \textit{Palpatine} is to take in a list of ballots.
Each ballot denotes the voter's list of preferences. We also should know a list
of all candidates because that cannot be derived from the list of preferences.
There is a scenario where a candidate gets not a single preference from not a
single voter. To express this, I have a file format, that is a modified version
of the one defined in \cite{stv_haskell}. Currently, their work requires that
candidates be declared in a Haskell GADT. One can imagine that the PBM Data
Description Language described in chapter 2.2 of the literature review could
parse and build such a data type, however I find that we can simply include the
total list of candidates at the top of the file. 

The file format consists of a list of the candidates (in parentheses to denote
that it is ont a ballot). A ballot is a list of preferences with the leftmost
preference representing the most desired candidate. Our ballots from our earlier
example in this data format is shown in Figure~\ref{stv_format}

\begin{figure}[ht!!!!!!!]
    \caption{Data format}
    \label{stv_format}
    \begin{lstlisting}
        (A,B,C):2
        [A,C]
        [A,B,C]
        [A,C,B]
        [B,A]
        [C,B,A]
    \end{lstlisting}
\end{figure}

We will be storing our total list of candidates in a length-indexed vector,
where candidate corresponds to the string name. We want to parse our ballots
such that as we elect a candidate, we remove preferences for that candidate from
all of our ballots. The best way to do that is to have ballots be a list of
\textit{Finite Numbers}, \texttt{Fin}. The type \texttt{Fin n} refers to a
natural number that is in the set of all natural numbers between \texttt{Zero}
and $n$. Each preference of type \texttt{Fin n} in our ballot corresponds to an
index in our \texttt{candidates} vector. As we elect a candidate and remove them
from our \texttt{candidates} vector, we reindex all of the ballots, changing the
\texttt{Fin}s to reflect the new indices of our now smaller vector of
candidates. How this high-level description looks like in terms of data types is
shown in Figure~\ref{stv_data_types}.

\begin{figure}[ht!!!!!!]
    \caption{Data types for \textit{Palpatine}}
    \label{stv_data_types}
    \begin{lstlisting}
        Candidate : Type
        Candidate = String

        Candidates : Nat -> Type
        Candidates n = Vect n Candidate
        
        VoteValue : Type
        VoteValue = Double

        -- New ballot type has a list of fins and
        -- a double as a pair. 
        total
        Ballot : Nat -> Type
        Ballot n = (List (Fin n), VoteValue)
    \end{lstlisting}
\end{figure}

We can parse a vector by using a dependent pair. We will not be able to derive
the type of the vector from elsewhere in our program given that it is data that
is derived through I/O. Thus, if we had our input file as a string, a
\texttt{getCandidates} function would have the following type signature:
\texttt{getCandidates : String -> (n ** Candidates n)}, where $n$ refers to the
number of candidates. We can then take the \texttt{Candidates n} produced and
parse for the ballots. The overall function \texttt{parseBallots} has the type
signature: \texttt{parseBallots : String -> Candidates n -> List \$ Ballot n}.

Thus, we generate a \texttt{List \$ Ballot n} containing the ballots which
reflect user preferences and a \texttt{Candidates n} which is a length-indexed
vector of candidates. 

\section{Counting}

\textit{This section is currently rough. It is an outline of existing and
planned work. It is mostly based on an outline of the STV counter provided in}
\cite{stv_haskell}.


We store our vote count as a hashmap with Strings as Keys representing each one
of our candidates and \texttt{VoteValue} (Doubles) as the values. In other words, we define a type: 

\texttt{VoteCount : Type} \\
\texttt{VoteCount = SortedMap Candidate VoteValue}

We start by
defining a function to calculate the Droop Quota and the surplus transfer.
\cite{stv_haskell} separates the vote counting process for the STV system into
the following formal steps: 

\subsection{Start}
We start by calculating the Droop Quota with a function \texttt{droopQuota : Nat
-> Nat -> Nat}. We then instantiate our VoteCount with a function
\texttt{instantiateVc : Candidates n -> VoteCount}, that simply instantiates a
VoteCount Map with all votecounts set at zero. 

\subsection{Count}



\subsection{Elect}

\subsection{Ewin}

\subsection{Hwin}

\subsection{Transfer}

\subsection{Elim}

List of potential improvements(?)
\begin{enumerate}
    \item Type level guarantees of correctness from the Droop Quota and Surplus Quota functions?
    \item Using type level rationals instead?
\end{enumerate}