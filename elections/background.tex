\chapter{Relevant Background on the Single Transferrable Vote}

Having provided a grounding in dependently-typed programming, I now move to the
voting system that I take an interest in. I take an interest in the Single
Transferable Vote (STV) system that is used to count voter preferences in many
developed democracies. STV is used in Ireland, Australia, New Zealand, and lower
level elections across the United Kingdom and the United States. Many variants
of the system exist and I concern myself with the method of STV currently used
to count votes in the Australian Senate. 

\section{A sample STV election}
\textit{This is an application of a description of the Australian Senate STV Counting Process as described by the Australian Electoral Commision.}

In order to demonstrate how the STV system works, it would help to run a sample
election, provided below. This sample election data is not original and obtained
from prior literature for an STV vote counter in Haskell and Coq
\cite{stv_haskell}. Suppose we have three candidates, $[A, B, C]$, running for
two available seats. Now suppose that we have 5 voters who each have sent in
preferences to be counted. The preferences are shown in
Figure~\ref{sample_election}. Each voter's preferences are referenced by a list
where the leftmost element represents the most preferred candidate and the
rightmost candidate presents the voter's least preferred. There is no obligation
to number all candidates. 


\begin{figure}[ht!!!!!!!!]
    \caption{Ballots for a sample election to be run. }
    \label{sample_election}
    \begin{lstlisting}
        [A,C]
        [A,B,C]
        [A,C,B]
        [B,A]
        [C,B,A]
    \end{lstlisting}
\end{figure}

The first thing to calculate is the \textit{Droop Quota}. This represents the voting `score' a candidate must have before they are elected. The Droop Quota is calculated in the Australian Senate by a formula shown below. 

$$
S = floor(\frac{b}{n + 1}) + 1
$$

In this formula, $b$ refers to the number of ballots, $n$ refers to the number
of seats available, and the \textit{floor} operation refers to rounding the
result down. The Droop Quota of our sample election is, according to this formula, 

$$
S = floor(\frac{5}{2 + 1}) + 1 = floor(\frac{5}{3}) + 1 = 1 + 1 = 2
$$

Now, we assign a \textit{weight} to each ballot. Our ballots are now: 

\begin{figure}[ht!!!!!!!!]
    \caption{Ballots with initial score.}
    \label{sample_election2}
    \begin{lstlisting}
        ([A,C], 1)
        ([A,B,C], 1)
        ([A,C,B], 1)
        ([B,A], 1)
        ([C,B,A], 1)
    \end{lstlisting}
\end{figure}

We now start by counting everyone's first preferences. $A$ receives 3 ballots
each with the weight of 1.0. Thus, $A$ has reached the Droop Quota and $A$ is
elected. No one else has hit the Droop Quota yet. Now, we transfer the surplus.
We remove the ``head'' from all the ballots that preferenced A and we transfer
the surplus votes. The new value of the votes is calculated as follows: 

$$
v = s/n
$$

where $v$ is the surplus transfer value, $s$ is the surplus ballots and $n$ is
the total number of ballots that preferenced the candidate first. By applying
this formula, the Droop Quota is 2, there are 3 votes for $A$ and so there is a
surplus of 1. $B, C$ each received one vote. This means the surplus is expressed
by $\frac{1}{3}$. In other words, our ballots now look like: 

\begin{figure}[ht!!!!!!!!]
    \caption{Ballots with initial score.}
    \label{sample_election2}
    \begin{lstlisting}
        ([C], 1/3)
        ([B,C], 1/3)
        ([C,B], 1/3)
        ([B], 1)
        ([C,B], 1)

        elected: [A]

        scores: 
        (B, 1)
        (C, 1)
    \end{lstlisting}
\end{figure}

We then immediately start counting the surplus transfer from our elected candidates. In other words, C gets an increase in score of $\frac{2}{3}$ and $B$ gets an increase in score of $\frac{1}{3}$. The scores of these ballots remain the same. 

\begin{figure}[ht!!!!!!!!]
    \caption{Ballots with initial score.}
    \label{sample_election3}
    \begin{lstlisting}
        ([], 1/3)
        ([C], 1/3)
        ([B], 1/3)
        ([B], 1)
        ([C,B], 1)

        elected: [A]

        scores: 
        (B, 4/3)
        (C, 5/3)
    \end{lstlisting}
\end{figure}

Now, given we have exhausted the surplus of elected candidates and there is
still one vacancy available, we remove the candidate with the lowest score,
which is $B$. We look at the votes that went to candidate $B$. There is one
exhausted ballot $[]$, and another ballot: $[C]$ with transfer value $1/3$. We
transfer this ballot over to $C$. We now have one candidate left in the running
and one seat open. This means that $C$ is elected by virtue of being the only
candidate left. 

The result of our election is thus, $[A,C]$ being elected to the senate.  

\section{Why STV?}
\textit{Coming in final draft of the thesis.} \\
In short, this will be a section on how STV allows someone to genuinely express
their candidate preferences in confidence that their vote will not be ``thrown
away.''

\section{Benefits and drawbacks of STV in summary}
\textit{Coming in final draft of the thesis.}

\section{Simplifying Assumptions}

There are a lot of simplifying assumptions in this text. We operate under the
assumption that a list of ballots that accurately describe the honest
preferences of the voters can be found. This in-and-of-itself means a lot of
simplifying assumptions which represent ongoing problems in online voting. If
paper ballots are employed, one must be certain that the paper ballots are read
into the data format correctly. This is fraught with potential for interference
at every step, although is possible. Similarly, if we are using electronic
ballots, we must be certain that the vote recorded in our data files matches the
vote inputted by the user. There is a wide body of literature on these issues
and they fall outside of the scope of this thesis. Work, for example, exists on
using formal methods to produce hardware-verifiable tamper-proof ballots with
PROM \cite{prom1}, \cite{prom2}. 

The question of whether ballots are \textit{honest} is a class of interesting
problems. In a traditional first-past-the-post, single-choiced ballot voting
system, one might decide to vote for a majority-party candidate with a larger
chance of winning instead of a candidate with a lower chance of winning. Third
party candidates become `vote spoilers' and voters are forced to engage in
`strategic voting'. They must choose a candidate that not necessarily best
reflects their political beliefs, but reflects the pragmatic reality of the
electoral system that they are presented with. Now, having introduced the
ability to rank, the question of whether a ballot \textit{honesty} takes on a
deeper meaning. Could someone maliciously preference the ballot in such a way
that the voting algorithm would favor a dishonest candidate preference? (e.g. to
elect candidate A, it is better to preference B, A, C instead of A, B, C). It
turns out that calculating dishonest preferences to game an STV election is an
NP-Complete problem \cite{strategic}. Calculating the margin of votes require to
game an STV election is also NP-Complete \cite{margin_of_victory}. 

I proceed with these assumptions in mind, that someone running a piece of ballot
counting software possesses a list of honest preferences that are protected from
tampering end-to-end. I acknowledge that there are significant challenges to
guarantee each one of these simplifications, but they are outside the scope of
my thesis. 
\section{Parsing}

Having now discussed how Palpatine calculates the results of an STV election, we
now move to parsing data files to build the values required as parameters for an
STV election. Our file format is a variant of the one defined in
\cite{stv_haskell}. The file format consists of a list of the candidates on the
first line, followed by a colon and a number denoting the number of seats to
elect. A ballot is a list of preferences with the leftmost preference
representing the most desired candidate. The remainder of our file after our
first line consists of a list of ballot preferences with each line denoting a
separate ballot. Our ballots from our earlier example in this data format is
shown in Figure~\ref{stv_format}

\begin{figure}[ht!!!!!!!]
    \caption{Data format}
    \label{stv_format}
    \begin{lstlisting}
        [A,B,C]:2
        [A,C]
        [A,B,C]
        [A,C,B]
        [B,A]
        [C,B,A]
    \end{lstlisting}
\end{figure}

In order to read this file format, we must first read the first line, building
the total length-indexed vector of the number of candidates remaining and the
number of seats remaining. We then read each ballot, building a list of ballots
and indexing each according to the length-indexed vector of candidates
remaining. We'll go through reading the first line as there is interesting
information around verification and types present, but we assume that parsing
the rest of the ballots would be something easily accomplished by a functional
programmer. 

\subsection{Reading Line 1}

\begin{figure}[ht!!!!!!!]
    \caption{Data format}
    \label{read_first_line_code}
    \begin{lstlisting}
        ExVect : Type -> Type
        ExVect t = (n ** Vect n t)

        total
        readFirstLine : String -> Maybe (ExVect Candidate, Nat)
        readFirstLine input = do
            let lines = splitToLines input
            firstLine <- head' lines
            let splitted =split (== ':') firstLine
            strCand <- head' splitted
            strNum <- last' splitted
            listCand <- parseList strCand
            let cands = map (\x => MkCandidate x 0) listCand
            seats <- parsePositive strNum
            pure $ ((toVec cands), cast seats)
    \end{lstlisting}
\end{figure}

The code for \texttt{readFirstLine} is given in
Figure~\ref{read_first_line_code}. The \texttt{readFirstLine} function takes in
a string which contains the contents of our data file and it outputs two items
contained by the first line. The first is an \texttt{ExVect} containing all the
candidates possible, and the second is a natural number that denotes the number
of seats available to elect. 

Something important to note here is the \texttt{ExVect} data type. We discussed
the concept of \textit{Dependent Pairs} earlier on in the literature review
section of this thesis in \autoref{pbm_review}. The \texttt{ExVect} is a
dependent pair between a natural number on the left and a length-indexed vector
on the right. That is, the vector on the right must have the length $n$, the
same length as the value of the natural number on the left. If we don't know the
length of a vector in advance, the existential vector, or \texttt{ExVect} is a
way to create a vector wrapped inside of a dependent pair. We can pattern match
on the an \texttt{ExVect} to remove the length-indexed vector wrapped inside of
it later. \texttt{ExVect}s are isomorphic to lists. 

All of these operations take place in the syntactic sugar of \textit{do}
notation afforded to us by the \texttt{maybe} monad. That is, if a single one of
these operations fails, our whole function will return \texttt{Nothing}. 

\section{Running Palpatine}

\begin{figure}[ht!!!!!!!]
    \caption{Data format}
    \label{main_code}
    \begin{lstlisting}
        total
        runElection : String -> IO ()
        runElection fileName = do
            Right str <- readFile fileName
            | Left err => putStrLn "ERROR: ReadFile Failed."
            case readFirstLine str of
                Just ((p ** cands), seats) => do 
                    let ballots = readBallots str cands
                    let dq = (droopQuota (length ballots) (cast seats))
                    case stv
                        ( makeElection 
                          dq
                          seats
                          ballots
                          cands
                          emptyResults
                        ) of
                        e@(_,_,_,_,results) => do
                            putStrLn "Done running the election. The results are:"
                            putStrLn $ show results
        Nothing => putStrLn "Parse error."
    \end{lstlisting}
\end{figure}

Palpatine takes in a filename and calls the \texttt{readFile} function in the
Idris standard library, taking in a string containing the file name and
producing an \texttt{IO String}. We then want the candidates and the ballots
from the first line. To produce this, we call our \texttt{readLine} function
which produces an ExVect of candidates, and a natural number denoting the number
of seats encapsulated inside of the Maybe monad. If parsing line 1 does not
succeed, producing a value \texttt{Nothing}, we simply error. If it does, we
pattern match on the ExVect that is produced, extracting the length-indexed
vector that is produced and the natural number of seats. We then run
\texttt{readBallots} with the length-indexed vector we just extracted from the
\texttt{ExVect}. We now have enough information to invoke our STV counter. The
code for \texttt{runElection} is given in Figure ~\ref{main_code}. 