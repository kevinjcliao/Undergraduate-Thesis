\chapter{Relevant Background on the Single Transferrable Vote}

Having provided a grounding in dependently-typed programming, I now move to the
voting system that I take an interest in. I take an interest in the Single
Transferable Vote (STV) system that is used to count voter preferences in many
developed democracies. STV is used in Ireland, Australia, New Zealand, and lower
level elections across the United Kingdom and the United States. Many variants
of the system exist and I concern myself with the method of STV currently used
to count votes in the Australian Senate. In this section, I will give an outline
of how STV works, provide optional pseudocode for the algorithm, and then run a
sample STV election to demonstrate how it works. 

\section{How STV works}

The Single Transferrable Vote works on a list of ballots each containing the
ranked choices (preferences) of a constituent. We start by calculating the Droop
Quota, given by the formula below, where $b$ represents the number of ballots
and $n$ represents the number of seats to elect. The Droop Quota, named after
English Lawyer and Mathematician Henry Droop, represents the minimum `score' of
a candidate required for them to be given a seat in the election
\cite{henry_droop}. The Droop Quota as a value represents the smallest number
such that there can be no more candidates that reach the quota than the number
of seats to be filled. If a candidate possesses a score greater than or equal to
the Droop Quota, we elect them. 

$$
DQ = floor(\frac{b}{n + 1}) + 1
$$

STV starts by counting all the first preferences of the ballots. In the
beginning, each candidate is assumed to have a `score' of zero. Each ballot has
a `weight', a value by which the `score' for the candidate preferenced first by
it is incremented on each count. Our vote counter then sees if any of the
candidates have a score higher than the Droop Quota. If any of them have a score
higher, we choose the candidate with the highest score and elect them. If not,
we eliminate the candidate with the lowest vote score. If a candidate is
elected, the weight of each ballot that voted for that candidate is changed to a
new transfer value. The transfer value is calculated by the formula: $v = s/n$.
In this formula, $s$ represents the surplus score (score - Droop Quota) and $n$
represents the the total number of ballots that preferenced the candidate. After
electing or eliminating a candidate, we then repeat again, counting each
ballot's next preference. If all the preferences of a ballot have been counted,
we just ignore the ballot. Ballots with no preferences are said to be
\textit{exhausted}. By iterating like this, eventually, the number of candidates
required to fill the seats will be elected. Pseudocode for single-transferrable
vote is given in Algorithm~\ref{imperative_stv}\footnote{This pseudocode is
adapted from a Github gist found here:\\
https://gist.github.com/jerinphilip/749393a53c0a6442e797cbba460ba989}


\begin{algorithm}
    \SetKwFunction{transferDown}{transferDown}
    \SetKwProg{Fn}{Function}{}{}
    \Fn{\transferDown{candidate}}{
        votes = $cand_{score}$\\
        excess = votes - droopQuota\\
        transferValue = excess/votes\\
        \For{ballot in ballots}{
            \If{ballot first preference is for candidate}{
                $ballot_{value}$ = transferValue
            }
            removeCandFromBallot
        }
    }
    \SetKwFunction{transferUp}{transferUp}
    \SetKwProg{Fn}{Function}{}{}
    \Fn{\transferUp{candidate}}{
        \For{ballot in ballots}{
            remove candidate from ballot
        }
    }
    \SetKwFunction{stv}{stv}
    \SetKwProg{Fn}{Function}{}{}
    \Fn{\stv{ballots, seats, candidates}}{
        let $dq$ = droopQuota\\
        let $vc$ denote the total vote score of each candidate where $vc_i$ is
        the score for candidate $i$\\
        let elected = empty list \\
        // Now we begin the first count. \\
        \For{each ballot b in ballots}{
            let $i$ = first preference of b\\
            $vc_i$ = $vc_i$ + 1\\
        }
        \While{ len(elected) < seats}{
            countNextBallots\\
            leader = getMaxCandScore(vc)\\
            \eIf{leader has higher score than droop Quota}{
                elected.append(leader)\\
                candidates.remove(leader)\\
                transferDown(leader)\\
            }{
                loser = getMinCandScore\\
                candidates.remove(loser)\\
                transferUp(loser)
            }
        }
        \Return{elected}
    }
    \caption{Imperative pseudocode for single-transferrable vote.}
    \label{imperative_stv}
\end{algorithm}


\section{A sample STV election}
\textit{This is an application of a description of the Australian Senate STV
Counting Process as described by the Australian Electoral Commision. The AEC
writeup is available
\href{https://www.aec.gov.au/Voting/counting/senate_count.htm}{here}}

In order to demonstrate how the STV system works, it would help to run a sample
election, provided below. This sample election data is not original and obtained
from prior literature for an STV vote counter in Haskell and Coq
\cite{stv_haskell}. Suppose we have three candidates, $[A, B, C]$, running for
two available seats. Now suppose that we have 5 voters who each have sent in
preferences to be counted. The preferences are shown in
Figure~\ref{sample_election}. Each voter's preferences are referenced by a list
where the leftmost element represents the most preferred candidate and the
rightmost candidate presents the voter's least preferred. There is no obligation
to number all candidates. 


\begin{figure}[ht!!!!!!!!]
    \caption{Ballots for a sample election to be run. }
    \label{sample_election}
    \begin{lstlisting}
        [A,C]
        [A,B,C]
        [A,C,B]
        [B,A]
        [C,B,A]
    \end{lstlisting}
\end{figure}

The Droop Quota of our sample election is, according to this formula, 

$$
S = floor(\frac{5}{2 + 1}) + 1 = floor(\frac{5}{3}) + 1 = 1 + 1 = 2
$$

Now, we assign a weight to each ballot. In the beginning of a run of the STV
algorithm, each ballot is weighed at 1. Our ballots are now: 

\begin{figure}[ht!!!!!!!!]
    \caption{Ballots with initial weight.}
    \label{sample_election0}
    \begin{lstlisting}
        ([A,C], 1)
        ([A,B,C], 1)
        ([A,C,B], 1)
        ([B,A], 1)
        ([C,B,A], 1)
    \end{lstlisting}
\end{figure}

We now start by counting everyone's first preferences. $A$ receives 3 ballots
each with the weight of 1.0. Thus, $A$ has reached the Droop Quota and $A$ is
elected. No one else has hit the Droop Quota yet. Now, we transfer the surplus.
We remove the ``head'' from all the ballots that preferenced $A$ and we transfer
the surplus votes. The Droop Quota is 2, there are 3 votes for $A$ and so there
is a surplus of 1. This means the surplus is expressed by $\frac{1}{3}$. By
removing all instances of $A$ from the ballots, setting ballots that preferenced
$A$ first to the transfer value, and removing all the heads of the ballots, we
finish our initial count. The ballots after this count are shown in
Figure~\ref{sample_election1}. 

\begin{figure}[ht!!!!!!!!]
    \caption{Ballots with initial score.}
    \label{sample_election1}
    \begin{lstlisting}
        ([C], 1/3)   A has been removed as a first preference. 
        ([B,C], 1/3) A has been removed as a first preference. 
        ([C,B], 1/3) A has been removed as a first preference. 
        ([], 1)      B has been removed as a first preference. A has been elected and removed. 
        ([B], 1)     C has been removed as a first preference. A has been elected and removed. 
        
        elected: [A]
        
        scores: 
        (B, 1)
        (C, 1)
    \end{lstlisting}
\end{figure}

Now, we count again the first preferences of each ballot. We then remove each
ballot's first preference. This results in the following ballots, shown in
Figure~\ref{sample_election2}

\begin{figure}[ht!!!!!!!!]
    \caption{Ballots after second count}
    \label{sample_election2}
    \begin{lstlisting}
        ([], 1/3)  C has been removed as a first preference. 
        ([C], 1/3) B has been removed as a first preference. 
        ([B], 1/3) C has been removed as a first preference. 
        ([], 1)    Ballot was exhausted in last count. 
        ([], 1)    B has been removed as a first preference. 
        
        elected: [A]
        
        scores: 
        (B, 7/3) Obtained from the first preferences of ballot 2 and 5
        (C, 5/3) Obtained from the first preferences of Ballot 1 and 3. 
    \end{lstlisting}
\end{figure}

Now, we see that B has passed the Droop Quota and is elected. With no seats
left, C is eliminated \footnote{This is different from the result in Ghale et.
al. because they use a different Droop Quota formula.}.

\section{Simplifying Assumptions}

There are a lot of simplifying assumptions in this text. We operate under the
assumption that a list of ballots that accurately describe the honest
preferences of the voters can be found. This in-and-of-itself means a lot of
simplifying assumptions which represent ongoing problems in online voting. If
paper ballots are employed, one must be certain that the paper ballots are read
into the data format correctly. This is fraught with potential for interference
at every step, although is possible. Similarly, if we are using electronic
ballots, we must be certain that the vote recorded in our data files matches the
vote inputted by the user. There is a wide body of literature on these issues
and they fall outside of the scope of this thesis. Work, for example, exists on
using formal methods to produce hardware-verifiable tamper-proof ballots with
PROM \cite{prom1}, \cite{prom2}. 

The question of whether ballots are \textit{honest} is a class of interesting
problems. In a traditional first-past-the-post, single-choiced ballot voting
system, one might decide to vote for a majority-party candidate with a larger
chance of winning instead of a candidate with a lower chance of winning. Third
party candidates become `vote spoilers' and voters are forced to engage in
`strategic voting'. They must choose a candidate that not necessarily best
reflects their political beliefs, but reflects the pragmatic reality of the
electoral system that they are presented with. Now, having introduced the
ability to rank, the question of whether a ballot \textit{honesty} takes on a
deeper meaning. Could someone maliciously preference the ballot in such a way
that the voting algorithm would favor a dishonest candidate preference? (e.g. to
elect candidate A, it is better to preference B, A, C instead of A, B, C). It
turns out that calculating dishonest preferences to game an STV election is an
NP-Complete problem \cite{strategic}. Calculating the margin of votes require to
game an STV election is also NP-Complete \cite{margin_of_victory}. 

I proceed with these assumptions in mind, that someone running a piece of ballot
counting software possesses a list of honest preferences that are protected from
tampering end-to-end. I acknowledge that there are significant challenges to
guarantee each one of these simplifications, but they are outside the scope of
my thesis. 