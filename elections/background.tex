\chapter{Relevant Background on the Single Transferrable Vote}

Having provided a grounding in dependently-typed programming, I now move to the
voting system that I take an interest in. I take an interest in the Single
Transferable Vote (STV) system that is used to count voter preferences in many
developed democracies. STV is used in Ireland, Australia, New Zealand, and lower
level elections across the United Kingdom and the United States. Many variants
of the system exist and I concern myself with the method of STV currently used
to count votes in the Australian Senate. 

\section{How STV works}

At a high level, STV works by first calculating the `score' of each candidate
and what they've received, knows as the \textit{Droop Quota}. If a candidate has
a total tallied vote value of higher than the Droop Quota, the candidate is
considered elected. We start by exhaustively counting the first preference on
everyone's ballot. When we finish counting these preferences, we see if any
candidate has met the Droop Quota. If such a candidate has met the Droop Quota,
we calculate their surplus (the surplus number of votes they received over the
Droop Quota). We then redistribute the ballots that voted for this winning
candidate by removing the ballots' first preference and lowering its value. If
after such a redistribution has taken place, another candidate reaches the Droop
Quota, we recount their preferences and so on. This process continues until no
count of next preferences will bring another candidate over the Droop Quota. 

When we have counted each ballot and redistributed accordingly and no candidates
will be brought over the line through a redistribution, we now eliminate the
candidate with the lowest `score' and we redistribute ballots that have voted
for that candidate to their next preferences. This continues until a candidate
hits the droop quota, after which we reallocate their surplus again. 

There are two termination conditions for this algorithm. The first is that all
seats are filled and there are leftover candidates. In such a case, we just
accept that the election is complete and discard the leftover candidates. The
second is that there are the same number or less number of leftover candidates
than there are seats available. In that case, we elect the leftover candidates
and close the election. Pseudocode for this algorithm is given below. This
pseudocode is modified from pseudocode provided by Sellstr\"om and T\~onisson
\cite{stv_pseudocode}. 

\begin{algorithm}
    \SetKwFunction{transferDown}{transferDown}
    \SetKwProg{Fn}{Function}{}{}
    \Fn{\transferDown{candidate}}{
        votes = $cand_{score}$\\
        excess = votes - droopQuota\\
        transferValue = excess/votes\\
        \For{ballot in ballots}{
            \If{ballot first preference is for candidate}{
                $ballot_{value}$ = transferValue
            }
            removeCandFromBallot
        }
    }
    \SetKwFunction{transferUp}{transferUp}
    \SetKwProg{Fn}{Function}{}{}
    \Fn{\transferUp{candidate}}{
        \For{ballot in ballots}{
            remove candidate from ballot
        }
    }
    \SetKwFunction{stv}{stv}
    \SetKwProg{Fn}{Function}{}{}
    \Fn{\stv{ballots, seats, candidates}}{
        let $dq$ = droopQuota\\
        let $vc$ denote the total vote score of each candidate where $vc_i$ is
        the score for candidate $i$\\
        let elected = empty list \\
        // Now we begin the first count. \\
        \For{each ballot b in ballots}{
            let $i$ = first preference of b\\
            $vc_i$ = $vc_i$ + 1\\
        }
        \While{ len(elected) < seats}{
            countNextBallots\\
            leader = getMaxCandScore(vc)\\
            \eIf{leader has higher score than droop Quota}{
                elected.append(leader)\\
                candidates.remove(leader)\\
                transferDown(leader)\\
            }{
                loser = getMinCandScore\\
                candidates.remove(loser)\\
                transferUp(loser)
            }
        }
        \Return{elected}
    }
\end{algorithm}


\section{A sample STV election}
\textit{This is an application of a description of the Australian Senate STV Counting Process as described by the Australian Electoral Commision.}

In order to demonstrate how the STV system works, it would help to run a sample
election, provided below. This sample election data is not original and obtained
from prior literature for an STV vote counter in Haskell and Coq
\cite{stv_haskell}. Suppose we have three candidates, $[A, B, C]$, running for
two available seats. Now suppose that we have 5 voters who each have sent in
preferences to be counted. The preferences are shown in
Figure~\ref{sample_election}. Each voter's preferences are referenced by a list
where the leftmost element represents the most preferred candidate and the
rightmost candidate presents the voter's least preferred. There is no obligation
to number all candidates. 


\begin{figure}[ht!!!!!!!!]
    \caption{Ballots for a sample election to be run. }
    \label{sample_election}
    \begin{lstlisting}
        [A,C]
        [A,B,C]
        [A,C,B]
        [B,A]
        [C,B,A]
    \end{lstlisting}
\end{figure}

The first thing to calculate is the \textit{Droop Quota}. This represents the
voting `score' a candidate must have before they are elected. The Droop Quota is
calculated in the Australian Senate by a formula shown below. 

$$
S = floor(\frac{b}{n + 1}) + 1
$$

In this formula, $b$ refers to the number of ballots, $n$ refers to the number
of seats available, and the \textit{floor} operation refers to rounding the
result down. The Droop Quota of our sample election is, according to this formula, 

$$
S = floor(\frac{5}{2 + 1}) + 1 = floor(\frac{5}{3}) + 1 = 1 + 1 = 2
$$

Now, we assign a \textit{weight} to each ballot. Our ballots are now: 

\begin{figure}[ht!!!!!!!!]
    \caption{Ballots with initial score.}
    \label{sample_election2}
    \begin{lstlisting}
        ([A,C], 1)
        ([A,B,C], 1)
        ([A,C,B], 1)
        ([B,A], 1)
        ([C,B,A], 1)
    \end{lstlisting}
\end{figure}

We now start by counting everyone's first preferences. $A$ receives 3 ballots
each with the weight of 1.0. Thus, $A$ has reached the Droop Quota and $A$ is
elected. No one else has hit the Droop Quota yet. Now, we transfer the surplus.
We remove the ``head'' from all the ballots that preferenced A and we transfer
the surplus votes. The new value of the votes is calculated as follows: 

$$
v = s/n
$$

where $v$ is the surplus transfer value, $s$ is the surplus ballots and $n$ is
the total number of ballots that preferenced the candidate first. By applying
this formula, the Droop Quota is 2, there are 3 votes for $A$ and so there is a
surplus of 1. $B, C$ each received one vote. This means the surplus is expressed
by $\frac{1}{3}$. By removing all instances of A from the ballots, setting
ballots that preferenced A first to the transfer value, and removing all the
heads of the ballots.

\begin{figure}[ht!!!!!!!!]
    \caption{Ballots with initial score.}
    \label{sample_election2}
    \begin{lstlisting}
        ([C], 1/3)
        ([B,C], 1/3)
        ([C,B], 1/3)
        ([], 1)
        ([B], 1)
        
        elected: [A]
        
        scores: 
        (B, 1)
        (C, 1)
    \end{lstlisting}
\end{figure}

Now, we count again the first preferences of each ballot. This results in the
following ballots: 

\begin{figure}[ht!!!!!!!!]
    \caption{Ballots after second count}
    \label{sample_election2}
    \begin{lstlisting}
        ([], 1/3)
        ([C], 1/3)
        ([B], 1/3)
        ([], 1)
        ([], 1)
        
        elected: [A]
        
        scores: 
        (B, 7/3)
        (C, 5/3)
    \end{lstlisting}
\end{figure}

Now, we see that B has passed the droop quota and is elected. With no seats
left, C is eliminated. \footnote{This is different from the result in Ghale et.
al. because they use a different droop quota formula.}

\section{Why STV?}
\textit{Coming in final draft of the thesis.} \\
In short, this will be a section on how STV allows someone to genuinely express
their candidate preferences in confidence that their vote will not be ``thrown
away.''

\section{Benefits and drawbacks of STV in summary}
\textit{Coming in final draft of the thesis.}

\section{Simplifying Assumptions}

There are a lot of simplifying assumptions in this text. We operate under the
assumption that a list of ballots that describe the accurate preferences of the
voters can be found. This entails a lot of simplifying assumptions which
represent ongoing problems in voting. If paper ballots are employed, one must be
certain that the paper ballots are read into the data format correctly. This is
fraught with potential for interference at every step, although is possible.
Similarly, if we are using electronic ballots, we must be certain that the vote
recorded in our data files matches the vote input by the user. There is a wide
body of literature on these issues and they fall outside of the scope of this
thesis \cite{election_safety}. Work, for example, exists on using formal methods
to produce hardware-verifiable tamper-proof ballots with PROM
\cite{prom1,prom2}. 

I proceed with these assumptions in mind, that someone running a piece of ballot
counting software possesses a list of accurate preferences that are protected from
tampering end-to-end. I acknowledge that there are significant challenges to
guarantee each one of these simplifications, but they are outside the scope of
my thesis. 

\chapter{Introducing Palpatine: An STV Vote Counter in Idris.}
\section{Parsing}

At its core, the goal of \textit{Palpatine} is to take in a list of ballots.
Each ballot denotes the voter's list of preferences. We also should know a list
of all candidates because that cannot be derived from the list of preferences.
There is a scenario where a candidate gets not a single preference from not a
single voter. To express this, I have a file format, that is a modified version
of the one defined in \cite{stv_haskell}. Currently, their work requires that
candidates be declared in a Haskell GADT. One can imagine that the PBM Data
Description Language described in chapter 2.2 of the literature review could
parse and build such a data type, however I find that we can simply include the
total list of candidates at the top of the file. 

The file format consists of a list of the candidates (in parentheses to denote
that it is ont a ballot). A ballot is a list of preferences with the leftmost
preference representing the most desired candidate. Our ballots from our earlier
example in this data format is shown in Figure~\ref{stv_format}

\begin{figure}[ht!!!!!!!]
    \caption{Data format}
    \label{stv_format}
    \begin{lstlisting}
        (A,B,C):2
        [A,C]
        [A,B,C]
        [A,C,B]
        [B,A]
        [C,B,A]
    \end{lstlisting}
\end{figure}

We will be storing our total list of candidates in a length-indexed vector,
where candidate corresponds to the string name. We want to parse our ballots
such that as we elect a candidate, we remove preferences for that candidate from
all of our ballots. The best way to do that is to have ballots be a list of
\textit{Finite Numbers}, \texttt{Fin}. The type \texttt{Fin n} refers to a
natural number that is in the set of all natural numbers between \texttt{Zero}
and $n$. Each preference of type \texttt{Fin n} in our ballot corresponds to an
index in our \texttt{candidates} vector. As we elect a candidate and remove them
from our \texttt{candidates} vector, we reindex all of the ballots, changing the
\texttt{Fin}s to reflect the new indices of our now smaller vector of
candidates. How this high-level description looks like in terms of data types is
shown in Figure~\ref{stv_data_types}.

\begin{figure}[ht!!!!!!]
    \caption{Data types for \textit{Palpatine}}
    \label{stv_data_types}
    \begin{lstlisting}
        Candidate : Type
        Candidate = String

        Candidates : Nat -> Type
        Candidates n = Vect n Candidate
        
        VoteValue : Type
        VoteValue = Double

        -- New ballot type has a list of fins and
        -- a double as a pair. 
        total
        Ballot : Nat -> Type
        Ballot n = (List (Fin n), VoteValue)
    \end{lstlisting}
\end{figure}

We can parse a vector by using a dependent pair. We will not be able to derive
the type of the vector from elsewhere in our program given that it is data that
is derived through I/O. Thus, if we had our input file as a string, a
\texttt{getCandidates} function would have the following type signature:
\texttt{getCandidates : String -> (n ** Candidates n)}, where $n$ refers to the
number of candidates. We can then take the \texttt{Candidates n} produced and
parse for the ballots. The overall function \texttt{parseBallots} has the type
signature: \texttt{parseBallots : String -> Candidates n -> List \$ Ballot n}.

Thus, we generate a \texttt{List \$ Ballot n} containing the ballots which
reflect user preferences and a \texttt{Candidates n} which is a length-indexed
vector of candidates. 

\section{Counting}

\textit{This section is currently rough. It is an outline of existing and
planned work. It is mostly based on an outline of the STV counter provided in}
\cite{stv_haskell}.


We store our vote count as a hashmap with Strings as Keys representing each one
of our candidates and \texttt{VoteValue} (Doubles) as the values. In other words, we define a type: 

\texttt{VoteCount : Type} \\
\texttt{VoteCount = SortedMap Candidate VoteValue}

We start by
defining a function to calculate the Droop Quota and the surplus transfer.
\cite{stv_haskell} separates the vote counting process for the STV system into
the following formal steps: 

\subsection{Start}
We start by calculating the Droop Quota with a function \texttt{droopQuota : Nat
-> Nat -> Nat}. We then instantiate our VoteCount with a function
\texttt{instantiateVc : Candidates n -> VoteCount}, that simply instantiates a
VoteCount Map with all votecounts set at zero. 

\subsection{Count}



\subsection{Elect}

\subsection{Ewin}

\subsection{Hwin}

\subsection{Transfer}

\subsection{Elim}

List of potential improvements(?)
\begin{enumerate}
    \item Type level guarantees of correctness from the Droop Quota and Surplus Quota functions?
    \item Using type level rationals instead?
\end{enumerate}
\section{Counting}

\textit{This section is currently rough. It is an outline of existing and
planned work. It is mostly based on an outline of the STV counter provided in}
\cite{stv_haskell} and code that I've implemented. 


We store our vote count as a hashmap with Strings as Keys representing each one
of our candidates and \texttt{VoteValue} (Doubles) as the values. In other
words, we define a type: 

\texttt{VoteCount : Type} \\
\texttt{VoteCount = SortedMap Candidate VoteValue}

After we have a list of ballots, we will want to distribute ballots between each
candidate. As each candidate is eliminated, we redistribute their ballots. In
other words, we should probably maintain another data structure, a Map of
Candidates to a list of ballots that correspond to the votes for that candidate.

The type for this data structure looks like as follows: 

\texttt{BallotDist : Nat -> Type} \\
\texttt{BallotDist n = SortedMap Candidate (List \$ Ballot n)}

All in all, it suffices to say that at every point in time, we will maintain a
list of elected candidates, a VoteCount map, a length-indexed vector of the
number of $n$ candidates left, and the number of $s$ seats to be filled. So, we
can define a constructor for the overall state of our STV election called the
\texttt{Election}. \texttt{Election} takes in two type variables, $n$ for the
candidates left and $s$ for the number of seats still available. The code for
such a type looks like this: 

\texttt{-- A tuple that stores: Number of candidates, list of ballots, seats available.}
\texttt{Election : Nat -> Nat -> Type}
\texttt{Election n s = (Candidates n, BallotDist n, s, VoteCount)}

We start by defining a function to calculate the Droop Quota and the surplus
transfer. \cite{stv_haskell} separates the vote counting process for the STV
system into the following formal steps: 

\subsection{Start}
We start by calculating the Droop Quota with a function \texttt{droopQuota : Nat
-> Nat -> Nat}. We then instantiate our VoteCount with a function
\texttt{instantiateVc : Candidates n -> VoteCount}, that simply instantiates a
VoteCount Map with all votecounts set at zero. As a start, we execute the
\texttt{count} step defined below for all the ballots. 

\subsection{Count}

\texttt{Count} takes the the \texttt{List \$ Ballot n} and splits them up by
first preference candidates. We see each ballot's first preferences and divide
the ballots into different piles, creating a \texttt{BallotDist n}. We then
build our \texttt{Election n s}, since we now have the candidates, number of
seats, the list of ballots and the votecount all instantiated.  As a
precondition, we check to make sure that neither $n$ and $s$ are not zero, and
that $n$ is greater than $s$, since if there are less candidates than the number
of seats available, all candidates are by default, elected. 

\begin{figure}[ht!!!!!!!]
    \caption{Count function}
    \label{firstcount}
    \begin{lstlisting}
        ||| firstCount runs through the ballots for the first time. It then
        ||| inserts the value into the VoteCount SortedMap. 
        ||| addVote is sort of an insertion function into the VoteCount sortedMap
        ||| addVote : Fin n -> Candidates n -> VoteValue -> VoteCount -> VoteCount
        total
        firstCount : Candidates n -> List (Ballot n) -> VoteCount -> VoteCount
        firstCount cands [] vc = vc
        firstCount cands (bal :: rest) vc = 
            firstCount cands rest newVc where
            newVc : VoteCount
            newVc = case bal of
                ((cand :: _), val) => addVote cand cands val vc
                ([], _)            => vc
    \end{lstlisting}
\end{figure}


\subsection{Elect}

After a \texttt{count} takes place, we now reach the \texttt{Elect} stage. The
Elect stage iterates through the VoteCount and checks to see which candidates
have now hit the Droop Quota. Candidates that have hit the quota are outputted
in a List of Candidates. I'm still uncertain how we would develop a type
signature for this, as we are unsure how many candidates could be eliminated and
elected in any given count so we don't know to what extent we will be
deprecating the natural numbers in the type of \texttt{Election n s}. I do know
that some \texttt{Election} data type will be outputted and a list of
candidates. The candidates will be removed from the candidate vector in the
Election and they will be added to the list of elected candidates. 

\begin{figure}[ht!!!!!!!]
    \caption{elect candidates}
    \label{electcands}
    \begin{lstlisting}
        total
        getElectedCands : Candidates n -> VoteCount -> Int -> (p ** Candidates p)
        getElectedCands cands vc dq = filter isOverQuota cands where
            isOverQuota : Candidate -> Bool
            isOverQuota cand = case getVoteVal cand vc of 
                Just voteVal => voteVal >= cast dq
                Nothing => False
    \end{lstlisting}
\end{figure}

\subsection{Transfer}
After we declare a winner, we transfer surplus ballots. This involves
calculating the surplus transfer value and modifying all of the ballots to no
longer preference the candidates that were just eliminated by mapping to the new
candidate vector that's inputted. We then move them to the new appropriate
BallotDist key for their next preference. 

\subsection{Elim}
\texttt{elim : Election (S n) s -> Election n s}. We eliminate a candidate when
we've exhausted the possible preferences. We run through the VoteCount,
determine which candidate has the least amount of votes and we go through our
BallotDist and redistribute their ballots. We then remove them from the
length-indexed vector of candidates. 

\subsection{Complete}
The base case for our algorithm is when there are no seats left to be elected,
or there are no candidates left to be elected. The \texttt{complete} function
has the type signature \texttt{Election n s -> List Candidate}. Essentially, we
pattern match on the number of seats left and the number of available
candidates. If either are zero, we have hit a base case, and we simply return
the list of elected candidates. 

By running \texttt{count}, \texttt{elect}, \texttt{transfer} continuously until
we can no longer do so, and then running \texttt{elim} before iterating through
\texttt{count}, \texttt{elect}, \texttt{transfer}. I assert that this algorithm
must terminate. How such a termination will be proven will be discussed in the
next section.

\subsection{Final Algorithm for STV}

\begin{figure}[ht!!!!!!!]
    \caption{Final STV type signature. }
    \label{stv_idris}
    \begin{lstlisting}
        total
        ||| Running an STV election involves taking in the candidates, the seats, the
        ||| ballots and producing a list of candidates to take that seat. 
        stv : Candidates n 
            -> List $ Ballots n 
            -> (seats : Nat) 
            -> (Candidates seats, Candidates, n - seats)
    \end{lstlisting}
\end{figure}

The final code for STV guarantees means we are guaranteed that upon termination,
we will elect the correct number of seats to the senate and the sum of the
eliminated candidates and the elected candidates will equal the number of input
candidates. 

List of potential improvements(????)
\begin{enumerate}
    \item Type level guarantees of correctness from the Droop Quota and Surplus Quota functions?
    \item Using type level rationals instead?
    \item Could we enforce that these two lists of candidates are disjoint at the type level?
\end{enumerate}