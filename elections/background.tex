\chapter{Relevant Background on the Single Transferable Vote}

Having provided a grounding in dependently-typed programming, I now move to the
voting system that I take an interest in. I take an interest in the Single
Transferable Vote (STV) system that is used to count voter preferences in many
developed democracies. STV is used in national elections in Ireland, Australia,
New Zealand, and lower level elections across the United Kingdom and the United
States. Many variants of the system exist and I concern myself with the method
of STV currently used to count votes in the Australian Senate. In this section,
I will give an outline of how STV works, provide optional pseudocode for the
algorithm, and then run a sample STV election to demonstrate how it works. 

\section{How STV works}

The Single transferable Vote works on a list of ballots each containing the
ranked choices (preferences) of a constituent. We start by calculating the Droop
Quota, given by the formula below, where $b$ represents the number of ballots
and $n$ represents the number of seats to elect. The Droop Quota, named after
English Lawyer and Mathematician Henry Droop, represents the minimum
\textit{score} of a candidate required for them to be given a seat in the
election \cite{henry_droop}. The Droop Quota as a value represents the smallest
number such that there can be no more candidates that reach the quota than the
number of seats to be filled. If a candidate possesses a score greater than or
equal to the Droop Quota, we elect them. 

$$
DQ = \lfloor\frac{b}{n + 1}\rfloor + 1
$$

STV starts by counting all the first preferences of the ballots. In the
beginning, each candidate is assumed to have a score of zero. Each ballot has a
\textit{weight}, a value by which the score for the candidate preferenced first
by it is incremented on each count. Our vote counter then sees if any of the
candidates have a score higher than the Droop Quota. If any of them have a score
higher, we choose the candidate with the highest score and elect them. If not,
we eliminate the candidate with the lowest vote score. If a candidate is
elected, the weight of each ballot that voted for that candidate is changed to a
new transfer value. The transfer value is calculated by the formula: $v = s/n$.
In this formula, $s$ represents the surplus score (score - Droop Quota) and $n$
represents the the total number of ballots that preferenced the candidate. After
electing or eliminating a candidate, we then repeat again, counting each
ballot's next preference. If all the preferences of a ballot have been counted,
we just ignore the ballot. Ballots with no preferences left to be counted are
said to be \textit{exhausted}. By iterating like this, eventually, the number of
candidates required to fill the seats will be elected. Pseudocode for
single-transferable vote is given in
Algorithm~\ref{imperative_stv}\footnote{This pseudocode is adapted from a Github
gist found here:\\
https://gist.github.com/jerinphilip/749393a53c0a6442e797cbba460ba989 and while
it is not from a verified source, it appears to conform to my interpretation of
the AEC's description of STV which is why it was used. I'll comment more about
the lack of a formal specification in the conclusions of this thesis.}.
Understanding this pseudocode in full is not necessary to continue reading on
but it provides a reference for anyone who may be interested. 


\begin{algorithm}
    \SetKwFunction{transferDown}{transferDown}
    \SetKwProg{Fn}{Function}{}{}
    \Fn{\transferDown{candidate, ballots}}{
        votes = $candidate_{score}$\\
        excess = votes - droopQuota\\
        transferValue = excess/votes\\
        \For{ballot in ballots}{
            \If{ballot has voted for candidate}{
                $ballot_{value}$ = transferValue
            }
            remove candidate from ballot if present
        }
        \Return{ballots}
    }
    \SetKwFunction{transferUp}{transferUp}
    \SetKwProg{Fn}{Function}{}{}
    \Fn{\transferUp{candidate, ballots}}{
        \For{ballot in ballots}{
            remove candidate from ballot if present
        }
        \Return{ballots}
    }
    \SetKwFunction{countBallots}{countBallots}
    \SetKwProg{Fn}{Function}{}{}
    \Fn{\countBallots{candidates, vc, ballots}}{
        \For{each ballot b in ballots}{
            let $i$ = first preference of b\\
            $vc_i$ = $vc_i$ + $b_{weight}$\\
            note that $b$ has voted for $i$
        }
        \Return{(vc,ballots)}
    }
    \SetKwFunction{stv}{stv}
    \SetKwProg{Fn}{Function}{}{}
    \Fn{\stv{droopQuota, ballots, seats, candidates}}{
        let $vc$ denote the total vote score of all candidates, $vc_i$ is
        the score for candidate $i$\\
        let elected = empty list \\
        // Now we begin the first count. \\
        \While{len(elected) < seats}{
            vc, ballots = countBallots(candidates, vc, ballots)\\
            leader = candidate in $vc$ with the highest score \\
            \eIf{$leader_{score}$ >= droopQuota}{
                elected.append(leader)\\
                candidates.remove(leader)\\
                ballots = transferDown(leader, ballots)\\
            }{
                loser = getMinCandScore\\
                candidates.remove(loser)\\
                ballots = transferUp(loser, ballots)
            }
        }
        \Return{elected}
    }
    \caption{Imperative pseudocode for single-transferable vote.}
    \label{imperative_stv}
\end{algorithm}


\section{A sample STV election}
\textit{This is an application of a description of the Australian Senate STV
Counting Process as described by the Australian Electoral Commision. The AEC
writeup is available
\href{https://www.aec.gov.au/Voting/counting/senate_count.htm}{here}}

In order to demonstrate how the STV system works, it would help to run a sample
election, provided below. This sample election data is not original and obtained
from prior literature for an STV vote counter in Haskell and Coq
\cite{stv_haskell}. Suppose we have three candidates, $[A, B, C]$, running for
two available seats. Now suppose that we have 5 voters who each have sent in
preferences to be counted. The preferences are shown in
Figure~\ref{sample_election}. Each voter's preferences are referenced by a list
where the leftmost element represents the most preferred candidate and the
rightmost candidate represents the voter's least preferred. There is no
obligation to number all candidates. 


\begin{figure}[ht!!!!!!!!]
    \caption{Ballots for a sample election to be run. }
    \label{sample_election}
    \begin{lstlisting}
        [A,C]
        [A,B,C]
        [A,C,B]
        [B,A]
        [C,B,A]
    \end{lstlisting}
\end{figure}

The Droop Quota of our sample election is, according to this formula, 

$$
S = \lfloor\frac{5}{2 + 1}\rfloor + 1 = \lfloor\frac{5}{3}\rfloor + 1 = 1 + 1 = 2
$$

Now, we assign a weight to each ballot. In the beginning of a run of the STV
algorithm, each ballot is weighed at 1. Our ballots are now given in
Figure~\ref{sample_election0}

\begin{figure}[ht!!!!!!!!]
    \caption{Ballots with initial weight.}
    \label{sample_election0}
    \begin{lstlisting}
        ([A,C], 1)
        ([A,B,C], 1)
        ([A,C,B], 1)
        ([B,A], 1)
        ([C,B,A], 1)
    \end{lstlisting}
\end{figure}

We now start by counting everyone's first preferences. $A$ receives 3 ballots
each with the weight of 1.0. Thus, $A$ has reached the Droop Quota and $A$ is
elected. No one else has hit the Droop Quota yet. Now, we transfer the surplus.
We remove the ``head'' from all the ballots and we change the weight of any
ballot that voted for A to a new transfer value. The Droop Quota is 2. $A$ has a
score of 3 and so there is a surplus of 1. This means the surplus is expressed
by $\frac{1}{3}$. By removing all instances of $A$ from the ballots, setting
ballots that voted for $A$ to the transfer value, and removing all the heads of
the ballots, we finish our initial count. The ballots after this count are shown
in Figure~\ref{sample_election1}. 

\begin{figure}[ht!!!!!!!!]
    \caption{Ballots with initial score.}
    \label{sample_election1}
    \begin{lstlisting}
        ([C], 1/3)   A has been removed as a first preference. 
        ([B,C], 1/3) A has been removed as a first preference. 
        ([C,B], 1/3) A has been removed as a first preference. 
        ([], 1)      B has been removed as a first preference. A has been elected and removed. 
        ([B], 1)     C has been removed as a first preference. A has been elected and removed. 
        
        elected: [A]
        
        scores: 
        (B, 1)
        (C, 1)
    \end{lstlisting}
\end{figure}

Now, we count again the first preferences of each ballot. We then remove each
ballot's first preference. This results in the following ballots, shown in
Figure~\ref{sample_election2}

\begin{figure}[ht!!!!!!!!]
    \caption{Ballots after second count}
    \label{sample_election2}
    \begin{lstlisting}
        ([], 1/3)  C has been removed as a first preference. 
        ([C], 1/3) B has been removed as a first preference. 
        ([B], 1/3) C has been removed as a first preference. 
        ([], 1)    Ballot was exhausted in last count. 
        ([], 1)    B has been removed as a first preference. 
        
        elected: [A]
        
        scores: 
        (B, 7/3) Obtained from the first preferences of ballot 2 and 5
        (C, 5/3) Obtained from the first preferences of Ballot 1 and 3. 
    \end{lstlisting}
\end{figure}

Now, we see that B has passed the Droop Quota and is elected. With no seats
left, C is eliminated \footnote{This is different from the result in Ghale et.
al. because they use a different Droop Quota formula.}.

\section{Simplifying Assumptions}

There are a lot of simplifying assumptions in this text. We operate under the
assumption that a list of ballots that describe the accurate preferences of the
voters can be found. This entails a lot of simplifying assumptions which
represent ongoing problems in voting. If paper ballots are employed, one must be
certain that the paper ballots are read into the data format correctly. This is
fraught with potential for interference at every step, although is possible.
Similarly, if we are using electronic ballots, we must be certain that the vote
recorded in our data files matches the vote input by the user. There is a wide
body of literature on these issues and they fall outside of the scope of this
thesis \cite{election_safety}. Work, for example, exists on using formal methods
to produce hardware-verifiable tamper-proof ballots with PROM
\cite{prom1,prom2}. 

I proceed with these assumptions in mind, that someone running a piece of ballot
counting software possesses a list of accurate preferences that are protected from
tampering end-to-end. I acknowledge that there are significant challenges to
guarantee each one of these simplifications, but they are outside the scope of
my thesis. 