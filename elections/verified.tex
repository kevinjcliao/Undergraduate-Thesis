\section{What have we verified?}

Having gone through the trouble of building Palpatine, the question we
now ask is: by building it, what have we verified? What compile-time guarantees
does Palpatine give us? In this section, I assert that Palpatine has
guarantees of termination and of the cardinality of judgments and remaining
candidates. 

\subsection{Termination}

One of the advantages of using Idris is its built in \texttt{total} keyword. If
a function declaration is prepended with the \texttt{total} keyword, the Idris
compiler will check at compile-time for \textit{totality}. While a specific
definition for totality is given in Idris' documentation, all that matters is
that the totality checker verifies that the program will terminate in a finite
amount of time and that all possible input cases are covered by the compiler. If
compilation succeeds, my program becomes a proof of termination. Given that I
annotated my \texttt{stv} function with the keyword \texttt{total}, it follows
that because my code compiles, my STV counter must terminate and that all input
cases are covered. 

Palpatine takes in the file to read from a user-provided command line argument.
Reading these command-line arguments is currently \texttt{partial}. As such, I
cannot guarantee that all of my code terminates at the moment, but with the
exception of one IO operation, all other code currently has termination
guarantees. 

\subsection{Cardinality of Candidates and Judgment}

Let's look at the type signature for \texttt{stv} and the guarantees that it
gives us. The function has type signature: \texttt{stv : Election r j ->
Election Z (r + j)}. What that means is, to put it quite simply, if \texttt{stv}
compiles, it is guaranteed that invoking STV with a list of candidates of
length-indexed vector of length \texttt{r} and a vector of judgments of length
\texttt{j} will produce a length-indexed vector of judgments that is exactly \texttt{r}
longer than \texttt{j}. In addition, given that we call \texttt{stv} from our
\texttt{runElection} function with an element of type \texttt{Election r Z}, we
have guarantees that Palpatine, when run, will judge exactly \texttt{r}
candidates. This is because the output of \texttt{stv} will be \texttt{Election
Z (r + j)} and since \texttt{j} was zero when runElection called \texttt{stv},
we thus have compile-time guarantees that the output will be a value of type
\texttt{Election Z r}. 

Cardinality is not formal verification that every one of my \texttt{r}
candidates becomes judged. There is, after all, no guarantee that the \texttt{r}
candidates judged by my \texttt{stv} function are the same candidates that were
input. That being said, it is highly unlikely that this would be the case. Even
if the programmer of a STV vote counter was adversarial, this would be an easily
spotted and unideal vector of attack. Thus, I can say with a high-degree of
certainty that Palpatine will judge every candidate input in an election as
either elected or eliminated. 

\subsection{Future Work}

Having verified this, there are next steps we can take to further verify
correctness of our implementation of Palpatine. Perhaps if we separated out the
\texttt{Results} length-indexed vector into a \texttt{Eliminated} and
\texttt{Elected} length-indexed vector, we could stipulate in our type signature
that our output \texttt{Elected} candidates must contain the same number of
elements as the number of seats allocated. Since the seats allocated is not
statically known and is given to us at compile-time, this would be an instance
where we can use dependent types. Further, we have no current guarantees that
the current set of candidates and the set of elected candidates each contain
unique elements. 