\chapter{What have we verified?}

Having gone through the trouble of building Palpatine, the question we
now ask is: by building it, what have we verified? What compile-time guarantees
does Palpatine give us? In this section, I assert that Palpatine has
guarantees of termination and of the cardinality of judgments and remaining
candidates. 

\section{Termination}

One of the advantages of using Idris is its built in \texttt{total} keyword. If
a function declaration is prepended with the \texttt{total} keyword, the Idris
compiler will check at compile-time for \textit{totality}. 

This definition of Totality is taken from \cite{tdd_book}. 
\begin{enumerate}
    \item A total function must cover all inputs. That is, our patterns must
    match every element in the set denoted by the type of our input.
    \item A total function must be \textit{well-founded}. That is, for all
    recursive calls or mutually recursive calls, ``it must be shown that one of
    its arguments has decreased''. 
    \item Our function must not have any arguments or produce any arguments that
    are data types that are \textit{strictly positive}. A data type $D$ that is
    not strictly positive would have a constructor with $D$ to the left of any
    arrow. 
    \item Finally, a total function must not call any other non-total function.
    It obviously follows that calling any non-total code ruins totality. 
\end{enumerate}

To prove that Palpatine terminates, I can thus use Idris' built-in
totality checker. If my code compiles and all functions are annotated with the
\texttt{total} keyword, then it follows that all of my code must satisfy all
four of these conditions and thus it must terminate. 

\section{Cardinality of Candidates and Judgment}

Let's look at the type signature for \texttt{stv} and the guarantees that it
gives us. The function has type signature: \texttt{stv : Election r j ->
Election Z (r + j)}. What that means is, to put it quite simply, if \texttt{stv}
compiles, it is guaranteed that invoking STV with a list of candidates of
length-indexed vector of length $r$ and a vector of judgments of length $j$ will
produce a length-indexed vector of judgments that is exactly $r$ longer. 

\section{Future Work}

Having verified this, there are next steps we can take to further verify
correctness of our implementation of Palpatine. Perhaps if we separated out the
\texttt{Results} length-indexed vector into a \texttt{Eliminated} and
\texttt{Elected} length-indexed vector, we could stipulate in our type signature
that our output \texttt{Elected} candidates must contain the same number of
elements as the number of seats allocated. Since the seats allocated is not
statically known and is given to us at compile-time, this would be an instance
where we can use dependent types. Further, we have no current guarantees that
the current set of candidates and the set of elected candidates each contain
unique elements. To take this even further, work in that area could bring us to
verify that the candidates that are elected are the ones expected in a run of
the algorithm. 