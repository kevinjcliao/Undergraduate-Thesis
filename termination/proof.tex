\chapter{Proof of Termination}

One of the advantages of using Idris is its built in \texttt{total} keyword. If
a function declaration is prepended with the \texttt{total} keyword, the Idris
compiler will check at compile-time for \textit{totality}. 

This definition of Totality is taken from \cite{tdd_book}. 
\begin{enumerate}
    \item A total function must cover all inputs. That is, our patterns must
    match every element in the set denoted by the type of our input.
    \item A total function must be \textit{well-founded}. That is, for all
    recursive calls or mutually recursive calls, ``it must be shown that one of
    its arguments has decreased''. 
    \item Our function must not have any arguments or produce any arguments that
    are data types that are \textit{strictly positive}. A data type $D$ that is
    not strictly positive would have a constructor with $D$ to the left of any
    arrow. 
    \item Finally, a total function must not call any other non-total function.
    It obviously follows that calling any non-total code ruins totality. 
\end{enumerate}

To prove that \textit{Palpatine} terminates, I can thus use Idris' built-in
totality checker. If my code compiles and all functions are annotated with the
\texttt{total} keyword, then it follows that all of my code must satisfy all
four of these conditions and thus it must terminate. 