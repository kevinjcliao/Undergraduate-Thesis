Databases are one of the pillars of modern software. Databases are employed in
everything from social media software \cite{tao} to flight scheduling and
booking \cite{flights}. Different programming languages have varying interfaces
for querying an SQL database. Oftentimes, a query and response interface simply
asks a user to send and receive a SQL query and as a string and to process a
response given as a string. While this offers flexibility, it means that there
is no compile-time guarantee of correctness and that poorly written SQL queries
result in run-time crashes, rather than compile-time debugging. We would like to
be able to easily compose types from a set of values (database schema), which
would allow a programmer to easily compose type-safe queries with compile-time
verification of correctness \cite{power_of_pi}. Many SQL interfaces do runtime
type checking, meaning if we can avoid these type checks, there may be
performance gains. 

\begin{figure}[h]
    \caption{Declaration of schema. Idris adaption of code from Power of Pi \protect\cite{power_of_pi}}
    \label{schema}
    \begin{lstlisting}
        Attribute : Type}
        (String, U)}

        Schema : Type}
        List Attribute}

        Students : Schema}
        Students = (   ("name", STRING)}
                    :: ("id", VECT 6 CHAR)} 
                    :: ("classyear", VECT 4 CHAR)}
                    :: Nil)}

        data Row : Schema -> Type where
            EmptyRow : Row Nil
            ConsRow : el u -> Row s -> Row ((name, u) :: s)

        kevin : Row Students
        kevin = ConsRow "Kevin Jiah-Chih Liao"
            (ConsRow ('1'::'2'::'3'::'4'::'5'::'6'::Nil)
            (ConsRow ('2'::'0'::'1'::'8'::Nil) EmptyRow))
    \end{lstlisting}
\end{figure}

We can start by defining types to represent a database schema. A database table
is a row of elements that correspond to a declared schema. A schema is a list of
attributes that each element should have. An attribute is simply a column name
and the type of what the column contains. Thus, we can declare an attributes and
schemas as follows in Figure ~\ref{schema}. 


Where U refers to the universe that we built in an earlier example. Given that
we built a schema, we are now able to define a schema to hold students. Take,
for example, a schema that stores a student's name, student id, and class year.
Our next job is to be able to express a database table with a row of instances
of a schema. This is reflected in our declaration of the \texttt{Row} type,
which lets us join rows together, ending in an EmptyRow. An example
\texttt{kevin} is provided. 

\begin{figure}[ht!]
    \caption{Declaration of connect. Idris adaption of code from Power of Pi
    \protect\cite{power_of_pi}}
    \label{connect}
    \begin{lstlisting}
        -- Connect takes in a server name, a table name, 
        -- and a schema, and executes IO operations on that table. 
        -- This type ensures that upon connection, if successful, 
        -- the table on the server must have the same schema. 
        connect : String -> String -> (s : Schema) -> IO (Handle s)

        
        -- Only modeling a single 'read' operation. 
        data RA : Schema -> Type where
            Read : Handle s -> RA s
        
        -- Takes in a query written with our relational algebra and then sends back a table that corresponds to our schema. 
        toSQL : RA s -> IO (List (Row s))
    \end{lstlisting}
\end{figure}

Now that we've defined the schema and tables, it's time to show how we connect
and query the table. Usually this means that we have a function \texttt{connect}
that takes in a servername and tablename as strings, and a SQL query as a
string, before returning a string in an IO monad. Now that we've defined a
schema type, we can build a well-typed database connection that validates that
the database table we are connecting to is of the same schema as the schema we
are requesting. If connection succeeds, that means that all subsequent requests
with the \texttt{Handle schema} type returned by the connect function must be
safe, because we know that the schemas we are reading or writing from the table
must correspond to the schema in our code. This connect function is defined in
Figure ~\ref{connect}. 

Finally, we define a type "Relational Algebra", which contains the opperations
that we would perform in a database query. For the sake of brevity, we've 
limited it to \texttt{Read}. We then have a function that does an IO operation
on a database, taking in a query of type Relational Algebra, to return us
a database table of consisting of the results of our query. 

Thus, in this example, we've shown the potential for dependent types to build a
type safe database, where if the schemas match on connection, all subsequent
queries should be type safe. While there was some overhead in setting up the
types, once the database software is written, declaring a schema with elements
in our universe was pretty easy and I anticipate database queries not being
significantly more difficult to compose and write. Thus, this is an exciting
application of dependent types where programmers can potentially benefit from
high-level use of dependently typed software once the more complex moving parts
are written. 

