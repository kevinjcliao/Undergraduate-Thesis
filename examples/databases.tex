\textit{This section is a summary of one of the sections from Oury \& Swierstra's
"The Power of Pi" \cite{power_of_pi}}

Our second example is a type-safe relational algebra. Databases are one of the
pillars of modern software. Databases are employed in everything from social
media software \cite{tao} to flight scheduling and booking \cite{flights}.
Different programming languages have varying interfaces for querying an SQL
database. Oftentimes, a query and response interface simply asks a client to
send an SQL query as a string and to process a response given as a string. While
this offers flexibility, it means that there is no compile-time guarantee of
correctness. This means that poorly written SQL queries result in run-time
crashes, rather than compile-time debugging. We would like to be able to easily
compose types from database table schema, which would allow a programmer to
write type-safe queries with compile-time verification of correctness
\cite{power_of_pi}. Many SQL interfaces do runtime type checking, meaning if we
can avoid these type checks, there may be performance gains. 

\begin{table}[h]
    \centering
    \begin{tabular}{|c|c|c|}
        Name : String & ID : Vect 6 Char & ClassYear : Vect 4 Char \\
        Kevin Liao    & 123456           & 2018                    \\
        David Smith   & 234567           & 2018
    \end{tabular}
    \caption{Table of students.}
    \label{students}
\end{table}

\begin{table}[h]
    \centering
    \begin{tabular}{|c|c|}
        ClassName : String & ID : Vect 8 Char\\
        Intro to CS        & CMSCH105        \\
        Writing Sem        & WRPRH101
    \end{tabular}
    \caption{Table of classes.}
    \label{classes}
\end{table}

\begin{table}[h]
    \centering
    \begin{tabular}{|c|c|c|c|c|}
        Name : String & ID : Vect 6 Char & ClassYear : Vect 4 Char & ClassName :
        String & ID : Vect 8 Char\\
        Kevin Liao    & 123456           & 2018       & Intro to CS & CMSCH105 \\
        Kevin Liao    & 123456           & 2018       & Writing Sem & WRPRH101 \\
        David Smith   & 234567           & 2018       & Intro to CS & CMSCH105 \\
        David Smith   & 234567           & 2018       & Writing Sem & WRPRH101 \\
    \end{tabular}
    \caption{Comparison between Haskell type signatures and mathematical proofs to illustrate Curry-Howard Isomorphism}
    \label{cartesian_product}
\end{table}

In this example, we are going to build up a simple dependently-typed relational
algebra in Idris. This relational algebra is going to allow us to read from a
database and also do a cartesian product. Let's say, for example, that I am a
registrar at a college have a table of students and a table of classes, and I
want to enroll each student in all the classes in the class table. In other
words I would want to compute the cartesian product of these tables, making sure
that they contain disjoint schema. Such an operation is shown in the following
tables. Table~\ref{students} shows a table of students, Table~\ref{classes} is a
table of classes and Table~\ref{cartesian_product} is the cartesian product of
the two tables.

\begin{figure}[h]
    \caption{Declaration of schema. Idris adaption of code from Power of Pi \protect\cite{power_of_pi}}
    \label{schema}
    \begin{lstlisting}
        Attribute : Type
        (String, U)

        Schema : Type
        List Attribute

        Students : Schema
        Students = (   ("name", STRING)}
                    :: ("id", VECT 6 CHAR)} 
                    :: ("classyear", VECT 4 CHAR)}
                    :: Nil)

        data Row : Schema -> Type where
            EmptyRow : Row Nil
            ConsRow : el u -> Row s -> Row ((name, u) :: s)

        kevin : Row Students
        kevin = ConsRow "Kevin Jiah-Chih Liao"
            (ConsRow ('1'::'2'::'3'::'4'::'5'::'6'::Nil)
            (ConsRow ('2'::'0'::'1'::'8'::Nil) EmptyRow))
    \end{lstlisting}
\end{figure}

We can start by defining types to represent a database schema. A database table
refers to rows of data that correspond to a declared schema. A schema is a list
of attributes that each element should have. An attribute is simply a column
name and the type of what the column contains. Thus, we can declare an
attributes and schemas as follows in Figure~\ref{schema}, where U refers to the
universe that we built in an earlier example. 

Given that we built a \texttt{schema} type, we are now able to define a schema
to hold students. Take, for example, a schema that stores a student's name,
student id, and class year. Our next job is to be able to express a database
table with a row of instances of a schema. This is reflected in our declaration
of the \texttt{Row} type, which lets us join rows together, ending in an
EmptyRow. An example \texttt{kevin} is provided. 

\begin{figure}[ht!]
    \caption{Declaration of connect. Idris adaption of code from Power of Pi
    \protect\cite{power_of_pi}}
    \label{connect}
    \begin{lstlisting}
        -- Connect takes in a server name, a table name, 
        -- and a schema, and executes IO operations on that table. 
        -- This type ensures that upon connection, if successful, 
        -- the table on the server must have the same schema. 
        connect : String -> String -> (s : Schema) -> IO (Handle s)

        -- Tells us if two schemas are disjoint. 
        disjoint : Schema -> Schema -> Bool

        -- A simple proposition that stops any false proposition from compiling
        So Bool -> Type
        So True = Unit
        So False = Void

        -- Only modeling Read and Product RA expressions.
        data RA : Schema -> Type where
            Read : Handle s -> RA s
            Product : (So $ disjoint s s') => RA s -> RA s' -> RA (append s s')
        
        -- Takes in a query written with our relational algebra and then 
        -- sends back a table that corresponds to our schema. 
        toSQL : RA s -> IO (List (Row s))
    \end{lstlisting}
\end{figure}

Now that we've defined the schema and tables, it's time to show how we connect
and query the table. Usually this means that we have a function \texttt{connect}
that takes in a server name and table name as strings, and a SQL query as a
string, before returning a string in an IO monad. We can contrast that unsafe
kind of database access with the one we're building here. Since we've defined a
schema type, we can build a well-typed database connection that validates that
the database table we are connecting to is of the same schema as the schema we
are requesting. Our version of \texttt{connect} takes in the server name and
table name and a schema. If connection succeeds, that means that all subsequent
requests with the \texttt{Handle schema} type returned by the connect function
must be safe, because we know that the schemas we are reading or writing from
the table must correspond to the schema in our code. 

Finally, we define a type \texttt{RA} which denotes expressions in our
relational algebra. For the sake of brevity, we've limited it to \texttt{Read}
and \texttt{Product}. We then have a function \texttt{toSQL} that does an IO
operation on a database, taking in an expression in \texttt{RA}, to return us a
database table of consisting of the results of our query. 

\texttt{Product} computes the Cartesian product of two tables, producing a
larger table where each row in the first table maps to every row in the second
table. This operation only works if the two schemas are disjoint. While it is
trivial to write a function \texttt{disjoint} that checks to see if two schemas
are disjoint, we want to enforce this at the type level. Thus, we want the type
of \texttt{Product} to enforce that its two schemas are disjoint. In addition,
we have a simple proposition \texttt{So} that ensures that whatever proposition
is supplied to it is valid. This is another example of an application of
dependent types. Whether a \texttt{Product} expression compiles is dependent
upon a check on the two schema values to ensure that its disjoint. Once again,
values are creating tighter bounds at the type level. 

Let's say that I now have to do the registrar cartesian product that I described
in the beginning of this section. What I'd first do is make two calls to the
\texttt{connect} function. I will send the schema of my students table in one
call and the schema of the classes table in another. If both connections
succeed, I will be given two values of type \texttt{Handle Student},
\texttt{Handle Class} respectively, which give me the certainty that the table
contains data conforming to my specified schemas. I can then feed these two
\texttt{Handle} values into a \texttt{Read} expression, which will give me
values of table rows of type \texttt{RA Student} and \texttt{RA Class}. I can
then take the cartesian product. My \texttt{Product} expression first checks to
make sure that the \texttt{Student} schema and \texttt{Class} schema are
disjoint, which they are, before appending the \texttt{Class} schema to the
\texttt{Student} schema to make the final schema that describes the rows of data
held by the results of our cartesian product of two tables. 

The crux of how dependent types are applicable in this example lies in the
\texttt{connect} function. In this example, we can see how the type of the
result from our connection to the database table depends on the value of the
schema that we pass in. We pass a schema as value, validate that the relevant
table contains data conforming to this exact schema. Subsequent expressions in
our relational algebra are predicated on database connections, validated with
dependent types. In a way, \texttt{connect} elevates schema values to the type
level and subsequent relational algebra expressions like \texttt{Read} or
\texttt{Product} can then carry out modifications of this schema at the type
level. 

Thus, in this example, we've shown the potential for dependent types to build a
type safe database, where if the schemas match on connection, all subsequent
queries should be type safe. While there was some overhead in setting up the
types, once the database software is written, declaring a schema with elements
in our universe was pretty easy and database queries not significantly more
difficult to compose and write. Thus, this is an exciting application of
dependent types where programmers can potentially benefit from high-level use of
dependently typed software once the more complex moving parts are written. 