\textit{This section is a summary of Edwin Brady's "Programming and Reasoning
with Algebraic Effects and Dependent Types" \cite{algebraic}}

In functional programming, we want to isolate side effects as much as possible
to keep our code clear. In Haskell and many other programming languages, side
effects like IO, state, random number generation, etc. are handled by monads
\cite{realworldhaskell}\footnote{This part of the literature assumes that you
are familiar with Monads, which is how the Haskell programming language handles
side effects. For more information, I highly recommend reading O'Sullivan et.
al's Real World Haskell or other widely available tutorials on Monads in
Haskell.}. If we want to use several Monads at once (our code requires
simultaneous handling of different side effects), we are often required to use
monad transformers. While this approach works for programs that require one or
two transformations between monads, as we bring in more and more side effects,
the number of transformation monads we need to write increases quite quickly. 

Work exists to sidestep the problem of handling increasingly complex monadic
transformations by encoding algebraic effects as a domain-specific language in a
dependently typed programming language \cite{algebraic}. We can start by
defining an \texttt{EFFECT} type, as seen in Figure~\ref{effects_def}. This
\texttt{EFFECT} data type enumerates the different kinds of side effects that an
Effects DSL could need to handle. In order for a function to use our effects
DSL, it will have to be of type \texttt{Eff}, where \texttt{Eff} is a data
declaration where the 'execution context' m (optionally a Monad) is specified, a
list of side effects, and the program's return type. For example, a function
that of the execution context \texttt{IO} that throws side effects, does work on
STDIO, and maintains an integer state will look something like the function
\texttt{example} in Figure~\ref{effects_def}. 

The code in Figure~\ref{effects_def} provides an example definition for
\texttt{State}. 

\begin{figure}[ht!!!!]
    \caption{Definition of effect type}
    \label{effects_def}
    \begin{lstlisting}
        data EFFECT : Type where
            STATE     : Type -> EFFECT
            EXCEPTION : Type -> EFFECT
            FILEIO    : Type -> EFFECT
            STDIO     : EFFECT
            RND       : EFFECT
        
        data Eff : (m : Type -> Type) -> (es : List EFFECT) -> 
            (a : Type) -> Type
    
        example : Eff IO [EXCEPTION String, STDIO, STATE Int] ()


        data State : Effect where
            Get : State a a a
            Put : b -> State a b ()

        get : Eff m [STATE x] x
        get = mkEffect Get

        put : x -> Eff m [STATE x] ()
        put val = mkEffect (Put val)
    \end{lstlisting}
\end{figure}

\texttt{EFFECT} enumerates different side effects that we'd expect an effects
DSL to handle for us. We want to define an \texttt{Effect} type which we can
express our side effects in terms of, and also, a type class \texttt{Handler}
that allows us to run expressions in a certain computational context. The
definition of these two items are given in Figure~\ref{state_def}. Notice how
\texttt{Effect} is parameterized by an input type, output type, and type of the
computation. Handler allows us to specify some computational context and then
run relevant effects. These definitions will become more clear as we implement
a \texttt{State} effect. 

The example that we will be dealing with later uses the \texttt{STATE} effect,
so we can look at its implementation. We want \texttt{STATE} to have a
\texttt{get} and {put} operation, of which, we expect \texttt{get} to be able to
be called an unlimited amount of times without any impact on the type of state.
We would like \texttt{put} to potentially mutate the type of \texttt{STATE}. We
therefore, define our two effects in the data class, where Get does not modify
the resource held in \texttt{State} whatsoever, so its effect looks exactly
paramaters are all the same type. \texttt{Put}'s mutation takes in a value of
type \texttt{b} and changes the type held by \texttt{State}. As a put operation,
we expect nothing to be returned, so the type of the computation is \texttt{()}.

Functions with different side effects are invoked from an execution context
\texttt{Env}. The specifics of how execution contexts are defined are not
important, however, we can note that different effects require different
execution contexts. \texttt{STATE}, for example, is strictly locally confined
and thus its execution context does not matter. Meanwhile, \texttt{STDIO} does
IO operations, and therefore its execution context must be inside of the IO
Monad. We can specify a type class \texttt{Handler}. All effects should be
instances of this type class. \texttt{Handler} enforces that the environment we
invoke a function from is the correct one. In other words, if I attempted to
invoke a \texttt{runpure} function that had an \texttt{STDIO} side effect, the
program will not compile. 


\begin{figure}[ht!!!!]
    \caption{Declaring handlers for side effects. Taken from Brady's work.
    \protect\cite{algebraic}}
    \label{handlers}
    \begin{lstlisting}
        Effect : Type
        Effect = (res : Type) -> (res' : Type) -> (t : Type) -> Type

        class Handler (e : Effect) (m : Type -> Type) where
            handle : res -> (eff : e res res' t) -> (res' -> t -> m a) -> m add
        
        instance Handler State m where
            handle st Get     k = k st st
            handle st (Put n) k = k n ()
        
        instance Handler StdIO IO where
            handle () (PutStr s) k = do putStrs; k () ()
            handle () GetStr     k = do x <- getLine; k () x
    \end{lstlisting}
\end{figure}




We can now apply this small Effects DSL we have defined to work on some simple
programs where we need to maintain a side effect of some sort. I will provide an
example of a program where we tag each node of a binary tree with a unique ID. 

\begin{figure}[ht!!!!]
    \caption{Tagging a binary tree with integers. Taken from Brady's work.
    \protect\cite{algebraic}}
    \label{tag_def}
    \begin{lstlisting}
        -- Simple type def of binary tree in Idris
        data Tree a = Leaf
                    | Node (Tree a) a (Tree a)
        
        -- Takes in a tree and produces a tagged tree with
        -- State containing an integer passed inside of
        -- the function. 
        tag : Tree a -> Eff m [STATE Int] (Tree (Int, a))
        tag Leaf = return Leaf
        tag (Node l x r) = do
            l' <- tag l
            lbl <- get; put (lbl + 1)
            r' <- tag r
            return (Node l' (lbl, x) r')
        
        EffM :  (m   : Type -> Type) ->
                (es  : List EFFECT) ->
                (es' : List EFFECT) ->
                (a   : Type) -> Type
        
        runPure : Env id es -> EffM id es es' a -> a
        
        tagFrom : Int -> Tree a -> Tree (Int, a)
        tagFrom x t = runPure [x] (tag t)
    \end{lstlisting}
\end{figure}

This tagging program is a simple function that recurses through a tree. The
function \texttt{get}s from and \texttt{put}s to a state that is kept alive for
the duration of the program. Since this side effect is 'created' at function
call-time and 'destroyed' on termination, it makes sense to say that this
function can be invoked as a pure function. Thus, we have a function with
internal side effects that have been clearly specified. We can see here how
specifying side effects as a list of effects rather than with monad transformers
means that we can easily add or remove side effects as required. 

Isolating side effects is a form of best practice and doing so is not easy.
While monads provide a way to handle side effects, programs multiple side
effects often require the use of monad transformers, making a programmer more
likely to take coarse-grained shortcuts which avoid many of the benefits of
monads in the first place. Building an \texttt{Effects} DSL allows the
programmer to specify effects of a program that executes inside a single
execution context. This allows for multiple monadic side effects to be
effectively juggled by a programmer. 