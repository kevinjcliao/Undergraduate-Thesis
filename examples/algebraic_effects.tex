In functional programming, we want to isolate side effects as much as possible
to keep our code clear. In Haskell and many other programming languages, side
effects like IO, State, Random Number Generation, etc. are handled by Monads
<CITATION NEEDED>. If we want to use several Monads at once (our code requires
simultaneous handling of different side effects), we are often required to use
monad transformers. While this approach works for programs that require one or
two transformations between monads, as we bring in more and more side effects,
the number of transformation monads we need to write increases quite quickly. 

Work exists to sidestep the problem of handling increasingly complex monadic
transformations by encoding algebraic effects as a domain-specific language in a
dependently typed programming language \cite{algebraic}. We can start by
defining an \texttt{EFFECT} type, as seen in Figure ~\ref{effects_def}. In order
for a function to use our effects DSL, it will have to be of type \texttt{Eff},
where Eff is a data declaration where the 'execution context' m (optionally a
Monad) is specified, a list of side effects, and the program's return type. For
example, a function that of the execution context \texttt{IO} that throws side
effects, does work on STDIO, and maintains an integer state will look something
like the function \texttt{example} in Figure ~\ref{effects_def}. 

\begin{figure}[h]
    \caption{Definition of effect type}
    \label{effects_def}
    \begin{lstlisting}
        data EFFECT : Type where
            STATE     : Type -> EFFECT
            EXCEPTION : Type -> EFFECT
            FILEIO    : Type -> EFFECT
            STDIO     : EFFECT
            RND       : EFFECT
        
        data Eff : (m : Type -> Type) -> (es : List EFFECT) -> 
            (a : Type) -> Type
    
        example : Eff IO [EXCEPTION String, STDIO, STATE Int] ()
    \end{lstlisting}
\end{figure}

We can now apply this small Effects DSL we have defined to work on some simple
programs where we need to maintain a side effect of some sort. I will provide an
example of a program where we tag each node of a binary tree with a unique ID. 

\begin{figure}[h]
    \caption{Tagging a binary tree with integers. Taken from Brady's work.
    \protect\cite{algebraic}}
    \label{tag_def}
    \begin{lstlisting}
        -- Simple type def of binary tree in Idris
        data Tree a = Leaf
                    | Node (Tree a) a (Tree a)
        
        -- Takes in a tree and produces a tagged tree with
        -- State containing an integer passed inside of
        -- the function. 
        tag : Tree a -> Eff m [STATE Int] (Tree (Int, a))
        tag Leaf = return Leaf
        tag (Node l x r) = do
            l' <- tag l
            lbl <- get; put (lbl + 1)
            r' <- tag r
            return (Node l' (lbl, x) r')
        
        get : Eff m [STATE x] x
        
        put : x -> Eff m [STATE x] ()
        
        EffM :  (m   : Type -> Type) ->
                (es  : List EFFECT) ->
                (es' : List EFFECT) ->
                (a   : Type) -> Type
        
        runPure : Env id es -> EffM id es es' a -> a
        
        tagFrom : Int -> Tree a -> Tree (Int, a)
        tagFrom x t = runPure [x] (tag t)
    \end{lstlisting}
\end{figure}