Dependent type systems have potential applications in easily implementing domain
specific languages (DSLs). Cryptol, for example, is a domain-specific language
designed around cryptography \cite{cryptol_manual}. Problems inherent in
implementing a Cryptol compiler or interpreter can be solved through dependent
types \cite{power_of_pi}. Cryptol is a functional programming language with
advanced support for pattern matching. Since cryptography commonly requires
dealing with low-level bit manipulation, it follows that Cryptol is designed
around facilitating these operations and making them safe. A function that does
this sort of low-level manipulation is the \texttt{swab} function, which takes
in a 32-bit word and swaps the first two bytes \cite{cryptol_manual}.: 

$$
\texttt{swab :: Word 32 -> Word 32} \\
\texttt{swab [a b c d] = [b a c d]} $$

Ideally, a word would be represented by a vector of 32-bits. We would be able to
declare a pattern match with swab that looks similar to the declaration
presented by Oury and Swiestra above. How then does the compiler understand that
this pattern match declaration means we expect the input vector to be divided
into 4 separate vectors of 8 bits? This is where dependent types serve a
practical purpose. By specifying types that split the length of the vector up
into a multiple of two scalars, we can effectively implement this clever pattern
match, allowing for powerful pattern matching required by the Cryptol language
(\cite{power_of_pi}). 

\textit{More coming in final draft of literature review}