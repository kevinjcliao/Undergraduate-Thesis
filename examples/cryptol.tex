\textit{This section is a summary of one of the sections from Oury & Swierstra's
"Power of Pi" \cite{power_of_pi}}

Dependent type systems have potential applications in easily implementing domain
specific languages (DSLs). Cryptol, for example, is a domain-specific language
designed around cryptography \cite{cryptol_manual}. Problems inherent in
implementing a Cryptol compiler or interpreter can be solved through dependent
types \cite{power_of_pi}. Cryptol is a functional programming language with
advanced support for pattern matching. Since cryptography commonly requires
dealing with low-level bit manipulation, it follows that Cryptol is designed
around facilitating these operations and making them safe. A function that does
this sort of low-level manipulation is the \texttt{swab} function, which takes
in a 32-bit word and swaps the first two bytes \cite{cryptol_manual}.: 

\texttt{swab :: Word 32 -> Word 32}\\
\texttt{swab [a b c d] = [b a c d]}

Ideally, a word would be represented by a vector of 32-bits. We would be able to
declare a pattern match with swab that looks similar to the declaration
presented by Oury and Swiestra above. How then does the compiler understand that
this pattern match declaration means we expect the input vector to be divided
into 4 separate vectors of 8 bits? This is where dependent types serve a
practical purpose. By specifying types that split the length of the vector up
into a multiple of two scalars, we can effectively implement this clever pattern
match, allowing for powerful pattern matching required by the Cryptol language
\cite{power_of_pi}. The final result looks like Figure~\ref{swab}. 

\begin{figure}[ht!]
    \caption{Final swab function \protect\cite{power_of_pi}}
    \label{swab}
    \begin{lstlisting}
        -- We specify in our type that we are splitting the
        -- input word into 4 bytes. 
        swab : Word 32 -> Word 32
        swab xs with (splitView 8 4 xs)
        swab _ | Split [a, b, c, d] = concat [b, a, c, d]

        data SplitView : {n : Nat} -> (m : Nat) 
                        -> Vect (m * n) a -> Type where
        Split : (xss : Vect m (Vect n a)) -> SplitView m (concat xss)
        
        splitConcatLemma : (xs : Vect (m * n) a) 
                            -> concat (split n m xs) = xs
        
        split : (n : Nat) -> (m : Nat) -> Vect (m * n) a -> Vect m (Vect n a)


        splitView : (n, m : Nat) -> 
                    (xs : Vect (m * n) a) -> SplitView m xs
        splitView n m xs =
          let prf  = sym (splitConcatLemma xs {m = m} {n = n})
              view = Split (split n m xs) {n = n} in
          rewrite prf in view
    \end{lstlisting}
\end{figure}

Idris and other dependently typed programming languages provide a \texttt{with}
keyword that lets you pattern match on the results of applying the inputs of a
function to another function. In this example, we are pattern matching on the
results of feeding in the input word to a function called \texttt{splitview}
that splits our word into 8 bytes. There are some constraints on such a
function. We expect that \texttt{splitView 9 3} will be an illegal call because
\texttt{splitView} takes in a 32-bit word yet this splitView is splitting a word
of size 27 ($9 \times 3$). We'll spend the remainder of this section exploring
what goes into implementing \texttt{splitView}. 

\texttt{splitView}'s implementation is also shown in Figure~\ref{swab}. It is
predicated on the existence of some function \texttt{split}, that splits up an
input vector into appropriate-length pieces based on input. splitView is
necessary because we cannot simply write \texttt{swab} as a call to
\texttt{split}. If our function 'swab' were simply a call to split, there would
be a problem. Since our swab function is a concatenation of the results of
split, we have no way of guaranteeing to the compiler that the result length of
that would be the same length as the input list. In other words, we cannot
guarantee \texttt{concat (split n m xs) = xs}, where \texttt{xs} refers to the
input 32-bit word. 

Such a proof is provided by \texttt{splitConcatLemma}. The specifics of this
inductive proof are provided in \cite{power_of_pi}. The example is provided to
illustrate that the machinations to make difficult pattern matching possible
with dependent types are non-trivial. However, once these proofs are
implemented, someone writing future compilers can very easily draw on them
simply by calling the \texttt{splitView} function to allow for the definition of
complex pattern-matching. 

What have dependent types bought us here? We are able to specify the constraints
on the dimensions of splitting our word. We know that the goal of
\texttt{split} is to be able to separate our input word into separate sizes. One
might argue that this is unnecessary overhead to introduce for the limited
certainty that the constituencies we aim to divide our word into make sense
(8*4=32). However, it's important to note that \texttt{swab} is a simple
function in Cryptol, and more complex functions do exist where whether the
function is implemented correctly is known with less certainty. Dependent types
provide a way to split and manipulate low-level bits while guaranteeing safety
through tightness of types. Given that Cryptol is a language for low-level
cryptography, a domain where bugs are both common, and incredibly dangerous, the
advantage of code with tighter constraints on types becomes apparent. 