Work also exists to use dependent types in creating embedded data description
languages, which are languages where a programmer can describe the structure of
data and quickly generate a working parser \cite{power_of_pi}.  For example,
consider the portable bitmap (pbm) file format, which consists simply of the
characters ``P4'', followed by the dimensions of the image in pixels as $n, m$
integers separated by a space. After a newline, the image is described as a
string of $n\times m $ bits where 1 is black and 0 is white \cite{pmb_spec}. A
sample of a file that conforms to this specification is shown below where $n=3,
m=4$. 

If a parser were generated from a data description language, we expect the
parser to either return well-typed data (a vector of bits and the dimensions of
the image) or to signal that the data is not well-formed. In other words, if we
want to generate parsers through embedded data description languages, we could
specify the file format as a value. The type of the generated parser then, would
depend on the file format as a value, making this an appropriate area to apply
dependent types \cite{power_of_pi}. 

\begin{figure}[h]
    \centering
    \caption{Sample PBM image}
    \label{pbm_sample}
    \begin{lstlisting}
        P4 3 4
        0 1 0
        1 1 1
        1 1 1
        0 1 0
    \end{lstlisting}
\end{figure}

We can start by defining our \textit{universe} (see Figure
~\ref{universe})\footnote{\label{idris_source}Original implementation in
\cite{power_of_pi} was in Agda. An Idris implementation was sourced from
https://github.com/yurrriq/the-power-of-pi/blob/master/src/Data/Cryptol.lidr.
All Idris code in this section was adapted from this repository.}, which
contains all the types that our parser will be manipulating in some way. We also
define a function \textit{el}, which will take any value of type U and maps it
to an appropriate type. The combination of this data type declaration and this
\textit{el} function is a definition of the relevant universe for this problem
domain \cite{power_of_pi}. 

\begin{figure}[h]
    \caption{Universe declaration in Idris \protect\cite{power_of_pi}}
    \label{universe}
    \begin{lstlisting}
        data Bit : Type where 
            O : Bit 
            I : Bit

        data U : Type where
            STRING : U
            BOOL : U
            CHAR : U
            NAT : U
            VECT : Nat -> U -> U
        
        el : U -> Type
            el STRING     = String
            el BOOL       = Bool
            el CHAR       = Char
            el NAT        = Nat
            el (VECT n u) = Vect n (el u)
    \end{lstlisting}
\end{figure}

From here, we can define a \textit{Format} data type that enables us to describe
the format of our data. When sequencing formats, we want two binary operators
that either read or skip. To skip means to skip over the first parameter and
generate a type for the file format from the second parameter. To read means to
build a type from both parameters before moving on. We will need to define both
these operations, a base operation that gives us a type, a terminal, and
rejection if the input data is badly formed. We can declare such a type as
shown in Figure~\ref{formatDeclaration}. 

In this code, notice how \texttt{Fmt} lifts values of type \texttt{Format} to
the type level. Data that is badly formatted will be of type \texttt{Void}, a
type showing that it has zero inhabitants. \texttt{End} maps to the
\texttt{Unit} type, which is a data type denoting one inhabitant. \texttt{Base}
takes in a type at the value from our type universe and uses \texttt{el} to
bring it to the type level. 

The definition of \texttt{Read} is a little curious. The use of the $(**)$
operator deserves some comment. This is the Idris syntactic sugar for a
dependent pair. A simple example for what dependent pairs are is shown in
Figure~\ref{dependentPairExample}. The type for the item on the right depends on
the value of the item on the left. In this case, we are specifying that the list
on the right must contain $n$ items, where $n$ is the value of the natural
number on the left. A dependent pair (a : A ** P) means that the variable $a$ is
of type A and can also occur in the type P \cite{tdd_book}. 

In the case of \texttt{Read}, what we are specifying is that 

\begin{figure}[ht!!!!!!]
    \caption{Format data type in Idris \protect\cite{power_of_pi}}
    \label{formatDeclaration}
    \begin{lstlisting}
        data Format : Type where 
        Bad : Format 
        End : Format
        Base : U -> Format
        Skip : Format -> Format -> Format 
        Read : (f : Format) -> (Fmt f -> Format) -> Format

        Fmt : Format -> Type 
        Fmt Bad = Void 
        Fmt End = Unit 
        Fmt (Base u) = el u
        Fmt (Read f1 f2) = (x : Fmt f1 ** Fmt (f2 x)) 
        Fmt (Skip _ f) = Fmt f

        (>>) : Format -> Format -> Format f1 >> f2 = Skip f1 f2

        (>>=) : (f : Format) -> (Fmt f -> Format) -> Format x >>= f = Read x f
    \end{lstlisting}
\end{figure}

In the code of Figure~\ref{formatDeclaration} one thing noteworthy is the $(**)$
operator, which is Idris syntactic sugar for a dependent pair. For example, consider Figure~\ref{dependentPairExample}.
The example will only type check correctly iff the natural number present in the
first element of the pair is the same as the length of the list. 

Having specified a data type that lets us declare formats, we can then move on
to creating a specification. In Figure ~\ref{spec_declaration}, a format for the
PBM spec is provided. We are now able to write a parser that takes in a Format
as a data type, and then is able to parse files to generate well-typed data. See
Figure~\ref{parser}. This straightforward parsing code is aided by the types
that we declared before, skipping, reading, and terminating where required by
our file format specification. We can use Idris' REPL to see how the parser
deals with our PBM specification in Figure ~\ref{repl}. Here, we see that the
type signature of the function created by giving the parse function our pbm
specification is a function that takes in a list of bits and returns a matrix of
bits with sizes bound by the natural numbers that we first specified in the file
format. 

\begin{figure}[h]
    \caption{Example for Dependent Pairs taken from the Idris documentation.}
    \label{dependentPairExample}
    \begin{lstlisting}
        vec : (n : Nat ** Vect n Int)
        vec = (2 ** [3, 4])
    \end{lstlisting}
\end{figure}

\begin{figure}[h]
    \caption{Format declaration of PBM \protect\cite{power_of_pi}}
    \label{spec_declaration}
    \begin{lstlisting}
        pbm : Format 
        pbm = char 'P' >> char '4' >> char ' ' >> Base NAT
        >>= \n => char ' ' >> Base NAT >>= \m => char '\n' >> Base (VECT n (VECT
        m BIT)) >>= \bs => End
    \end{lstlisting}
\end{figure}


\begin{figure}[ht!!!!!]
    \caption{Parser declaration \protect\cite{power_of_pi}}
    \label{parser}
    \begin{lstlisting}
        parse : (f : Format) -> List Bit -> Maybe (Fmt f, List Bit)
        parse Bad bs       = Nothing
        parse End bs       = Just ((), bs)
        parse (Base u) bs  = read u bs
        parse (Plus f1 f2) bs with (parse f1 bs)
          | Just (x, cs)   = Just (Left x, cs)
          | Nothing with (parse f2 bs)
            | Just (y, ds) = Just (Right y, ds)
            | Nothing      = Nothing
        parse (Skip f1 f2) bs with (parse f1 bs)
          | Nothing        = Nothing
          | Just (_, cs)   = parse f2 cs
        parse (Read f1 f2) bs with (parse f1 bs)
          | Nothing        = Nothing
          | Just (x, cs) with (parse (f2 x) cs)
            | Nothing      = Nothing
            | Just (y, ds) = Just ((x ** y), ds)
    \end{lstlisting}
\end{figure}

\begin{figure}[ht]
    \caption{Putting the PBM spec into Idris' REPL}
    \label{repl}
    \begin{lstlisting}
        *Parser> :t parse pbm parse pbm : List Bit -> Maybe ( (x : Nat ** x1 :
        Nat ** x2 : Vect x (Vect x1 Bit) ** ()) , List Bit)
    \end{lstlisting}
\end{figure}

\subsection{Comments}

In this section, we show that dependent types allow us to create embedded data
description languages inside of a dependently typed language. We can then
generate well-typed, reliable parsers without having to rewrite a lot of code.
Thus, using dependently typed languages to write parsers with embedded data
description languages both demonstrates promise in brevity and also safety. 