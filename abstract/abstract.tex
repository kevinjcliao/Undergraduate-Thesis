
\addcontentsline{toc}{chapter}{Abstract}

\begin{abstract}

Programs crash. Static type systems serve to make the life of a programmer
easier by providing static compile-time guarantees of certain program behaviour,
guaranteeing that a function will map from a domain to a co-domain. A
dependently typed programming language gives a programmer even additional
expressive power, allowing types to be expressed relative to values given at
runtime. Thus, the co-domain of our function is dependent upon some value that is
provided at runtime. 

While research in dependent types has traditionally been in the domains of
theoretical mathematics, where researchers use dependently typed theorem provers
to write mathematical formulisms, dependent types have applications outside of
such a theoretical domain. In my work, I seek to summarize the existing body of
work around practical applications of dependent types. This review is a  tour of
a potential application of dependent types. My review explores the following
additional examples: 
\begin{enumerate}
  \item Complicated pattern matching in the Cryptol DSL for cryptographic
  applications. \cite{power_of_pi}
  \item Generating file format parsers from a data description language.
  \cite{power_of_pi}
  \item A relational database and algebra. \cite{power_of_pi,eisenberg2016}
  \item An alternative to monadic transformations with an algebraic effects DSL
  \cite{algebraic}
  \item Programming distributed systems with F* and dependent types.
  \cite{fstar_distributed_programming}
  \item A low-level domain-specific language demonstrating the applicability of
  dependent types to systems programming. \cite{idris_systems_programming}
\end{enumerate}

Through a review of the practical applications of dependent types in existing
and implemented functional programming languages, I demonstrate that while
dependent types are often thought of as far-off and theoretical, they are
currently available in advanced functional programming languages and serve
practical purposes. 

I set out to look for patterns that signal to a programmer that dependent types
would be particularly useful for their work. So far, I've found that the
problems that I'm exploring seem to be focused around two main themes: the
serializing and deserializing of data, and building domain-specific languages. 

\end{abstract}
